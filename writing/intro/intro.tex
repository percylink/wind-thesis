
\documentclass[12pt]{amsart}
\usepackage{geometry} % see geometry.pdf on how to lay out the page. There's lots.
\usepackage{datetime}
\geometry{a4paper} % or letter or a5paper or ... etc
% \geometry{landscape} % rotated page geometry

% See the ``Article customise'' template for come common customisations

\title{Wind chapter}
\author{Percy Link}
\date{\currenttime \ \today} % delete this line to display the current date

%%% BEGIN DOCUMENT
\begin{document}

\maketitle

\section{Introduction}

As illustrated in the previous chapter, soil moisture influences the near-surface atmosphere by controlling fluxes of heat at the land surface.  In this chapter, I investigate the effect of soil moisture on winds at a specific wind farm, the Solano Wind Project in the Sacramento-San Joaquin River Delta region of California.  I use a regional atmospheric model to test which regions' soil moisture have the greatest effect on winds at the Solano wind farm and to quantify the magnitude of the effect at different times of day across a range of soil moisture changes.  This study serves as a prototype for characterizing the importance of soil moisture information for wind forecasts at other wind farms.  A goal of this study is to demonstrate that accurate soil moisture information can improve wind forecasts, and that constraining the regions to which the wind forecast is most sensitive may help target soil moisture measurement efforts.

Accurate wind forecasts can reduce the cost of integrating wind energy into the electric grid on a large scale.  In order to reduce CO$_2$ emissions to the degree necessary to avert dangerous climate change [CITE IPCC], electric utilities will need to transition to non-fossil-fuel energy sources including a large fraction of wind energy [CITE Jacobson].  However, although wind energy has large peak generation potential, it is intermittent, and the instantaneous mismatch between wind generation and electric demand must be met with other generation.  In most utilities, allocations of electric generation are made a day ahead, based on forecasted demand and wind and solar supply.  At shorter lead times (hour-ahead to real-time), imbalances between the day-ahead allocations of generation (based on the day-ahead forecasts of wind, solar, and demand) and the actual net demand (demand minus wind and solar) must be met with, in the case of a shortfall, more expensive quick-startup generation, or in the case of excess generation, wasted energy resources.  These imbalance costs add significantly to the cost of wind energy (up to XX\%) [CITATIONS - Porter, Fabbri, Ummels].  Thus, improved accuracy of wind forecasts could reduce imbalance costs, making integration of wind into the electric grid more economically feasible.

Current state of the science in wind forecasting
\begin{itemize}
\item physical, statistical, and hybrid modeling methods
\item different forecasting companies and organizations have different models - give some examples, cite review papers (Costa, Foley, Lei, Monteiro, Porter, Wang
\item tests of model resolution and physical parameterizations, especially planetary boundary layer schemes [Draxl, Marjanovic], also large scale forcing (Carvalho)
\item sensitivity to terrain complexity (Marjanovic, Carvalho)
\item tests of model output statistics (MOS) algorithms to correct NWP forecast based on historical observations; in particular, tests of various machine learning algorithms [B\'edard, Ellis, Kusiak, Ortiz-Garc\'ia, Ranaboldo, Sweeney]
\item prediction of ramps, but mostly in a statistical sense, without explicit connection to the physical mechanisms driving the rapid increases or decreases in wind power [Carcangiu, Ellis]
\item ensembles of NWP runs, initialized with different random perturbations (?) (Deppe, Pinson
\end{itemize}

The influence of land surface energy fluxes on wind energy forecasts has not received much attention (with the exception of Marjanovic et al. XXXX who show that the forecast at a California wind farm is broadly sensitive to initial soil moisture, and WHARTON LITERATURE ON STABILITY, and WHARTON-MAXWELL ON PARFLOW.)  However, the effect of soil moisture on regional circulations has been recognized for several decades in the literature on atmospheric boundary layer processes and weather forecasting.  [OVERVIEW OF LITERATURE]
\begin{itemize}
\item Soil moisture heterogeneity can drive mesoscale circulations [Avissar lit, ] - preferred length scales, sensitivity to heat flux anomaly amplitude, dependence on synoptic conditions
\item Thermal contrast between land and ocean drives mesoscale circulations [CA sea breeze lit], and the strength of the thermal contrast depends in part on soil moisture because of its influence on land surface temperature [Physick, Miller]
\item Nikola's paper: soil moisture has strong affect on accuracy of wind forecast, both in magnitude and in timing of wind shifts (``ramps")
\item Stability matters: Wharton and Lundquist found that turbines extracted more power for the same nacelle wind speed in stably stratified conditions than in convective conditions.  This may depend on the site: others (XXX) have found that turbines extract less power in stable conditions than in convective conditions, for the same wind speed.  In either case, stability near the surface depends strongly on the surface energy balance, which in turn depends strongly on soil moisture.
\end{itemize}

Winds in the Solano area are strongly influenced by both the surrounding complex terrain and the contrast of land surface heating with the adjacent cool ocean.  Solano sits in a valley pass between the ocean and California's Central Valley, and onshore winds are channeled and accelerated through the pass; this topographic channeling also constrains the wind direction at low levels near Solano to remain near westerly.  The diurnal cycle of land surface heating drives a marked diurnal cycle in wind speed in the Solano area, with minimum wind speeds in the morning and maximum speeds in the late afternoon and evening.  Additionally, the strongest winds occur in the summer at Solano, in part due to generally calm synoptic conditions (created by the north Pacific summertime high pressure) that allow the strong surface temperature contrast between ocean and land to drive onshore flow. [CITE CA WIND CLIMATOLOGIES]  Because winds at Solano are generated partly by land surface heating contrasts, it is likely that changes/errors in soil moisture would create errors in wind forecasts by NWP models.

Increased measurement of soil moisture and improved integration of land surface heat flux measurements into forecast models could significantly improve wind forecast accuracy.  We approach the problem from the perspective of a wind farm operator who wants to improve wind energy forecasts by measuring soil moisture.  If we were to measure soil moisture, what locations would give the greatest forecast improvement?  At what spatial resolution should the measurements be made?  The purpose of this study is not to make specific predictions or match observations; rather, we seek to characterize the dependence of wind at this location on soil moisture, as a model for future studies in support of measurement and data assimilation at wind farms more broadly.

We seek to answer the following questions:
\begin{itemize}
\item On which regions' soil moisture do Solano winds depend?
\item What changes in the amplitude and timing of the diurnal cycle result from changes/errors in soil moisture?
\item At what time of day is the wind forecast most sensitive to the accuracy of soil moisture information?
\item What is the physical mechanism for soil moisture's influence on Solano winds?
\item How do wind forecast errors scale with soil moisture changes/errors?  Are there particular ranges of soil moisture where wind forecasts are particularly sensitive?
\item MOVE TO DISCUSSION? Does soil moisture contribute to the timing and/or magnitude of "ramps" (rapid changes in wind speed)?
\item How sensitive is the wind forecast to the areal coverage of soil moisture changes?
\end{itemize}

\end{document}