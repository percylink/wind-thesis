
\documentclass[12pt]{amsart}
\usepackage{geometry} % see geometry.pdf on how to lay out the page. There's lots.
\usepackage{datetime}
\geometry{a4paper} % or letter or a5paper or ... etc
% \geometry{landscape} % rotated page geometry

% See the ``Article customise'' template for come common customisations

\title{Wind chapter}
\author{Percy Link}
\date{\currenttime \ \today} % delete this line to display the current date

%%% BEGIN DOCUMENT
\begin{document}

\maketitle

\section{Introduction}

Segue from previous chapters:
\begin{itemize}
\item Flux of water from the land surface affects circulations on a regional scale (as shown in chapter on boundary layer)
\end{itemize}

Set up the problem of wind forecasting:
\begin{itemize}
\item wind is a variable source of energy
\item utility needs to plan for this variability, so that they have enough energy when wind is low but don't have too much extra when wind is high
\item having too much extra generation: paying for energy they don't need, and also emissions that aren't necessary
\item having too little generation when wind is low: what happens?  have to pay for expensive ramp-up of plants, or have to pay penalties?  (confirm this)
\item how better forecasting can help: more accurate bidding/committing to generation on the day ahead (or even on hour-ahead?), which can save XX amount of money; less need for generation plants spinning in reserve, which can save XX amount of emissions
\end{itemize}

How wind energy forecasting currently works
\begin{itemize}
\item physical, statistical, and hybrid modeling methods
\end{itemize}

Need for more physical/mechanistic understanding of what drives the wind dynamics.  Mechanistic understanding enables smarter statistical model-building, because you have a better sense of the relevant variables to put in the model.  Mechanisms will play out in different way in different locations... But this study is a prototype for how to parse important mechanisms...

This study focuses on the mechanism of land-atmosphere energy exchange as modulated by soil moisture.  Plenty of previous work has shown that soil moisture heterogeneity can drive mesoscale circulations (citations), but much of this literature focuses on the effect of heterogeneities on cloud formation and precipitation.  Here, we focus on the effect of such heterogeneities on surface winds and on wind predictability.  

We approach the problem from the perspective of a wind farm operator who wants to improve wind energy forecasts by measuring soil moisture.  If we were to measure soil moisture, what locations would give the greatest forecast improvement?  At what spatial resolution should the measurements be made?

State how this study is different from Avissar-esque work: focus on response of near-surface wind at a point (or fixed wind farm area) to soil moisture heterogeneities in different configurations around the point.  How do these soil moisture heterogeneities affect wind prediction accuracy, if soil moisture heterogeneities are not known and not included in model?  Particular times of day?  What types of errors?  (phase shift in onset of diurnal cycle, or change in speed, or new appearance of ``ramps")

State how this study is different from wind energy forecasting literature: (careful here - don't want to be too hard on a body of work with which I am not very familiar) - focus on physical mechanism; not trying to predict at a specific wind farm, in this case; instead, using a physically-based model to identify regions from which soil moisture information is particularly important for forecasts at a given location, and times of day and synoptic conditions when forecasts are particularly sensitive to soil moisture information.

\subsection{Literature background}

\subsection{WRF regional atmospheric model}

\end{document}