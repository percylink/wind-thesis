
%\documentclass[12pt]{amsart}
%\usepackage{geometry} % see geometry.pdf on how to lay out the page. There's lots.
%\usepackage{datetime}
%\usepackage{setspace}
%\doublespacing
%\geometry{a4paper} % or letter or a5paper or ... etc
%% \geometry{landscape} % rotated page geometry
%
%% See the ``Article customise'' template for come common customisations
%
%\title{Wind chapter}
%\author{Percy Link}
%\date{\currenttime \ \today} % delete this line to display the current date
%
%%%% BEGIN DOCUMENT
%\begin{document}
%
%\maketitle

\section{Results}

We first characterize the differences in Solano turbine-level wind resulting from the soil moisture tests (Section \ref{subsec:CharWindChanges}).  We then investigate the physical mechanism linking changes in soil moisture to changes in Solano wind timing and magnitude (Section \ref{subsec:PhysMech}).

\subsection{Characterization of Solano wind sensitivity to soil moisture}
\label{subsec:CharWindChanges}

\subsubsection{Regional sensitivity}

Before presenting the Solano winds, we quantify the synoptic and land-surface-heating forcing in each case, and we characterize the control case winds.  The synoptic forcing, measured by wind speed at XX m, varies little between the model runs (Figure XX), and synoptic winds are weak on XX days and strong on XX days.  Soil moisture declines by XX m$^3$/m$^3$ or less before being reset each midnight (Figure XX).  Land surface heating \textbf{DESCRIBE DIFFERENCES HERE}

\begin{figure}[here]
\includegraphics[width=1\textwidth]{img/solano_wind_wetbkd_dryrg_d02_level0.png}
\caption{Time series of wind speed magnitude at 60 m AGL for the d02 grid point nearest the Solano wind farm, for the wet background and dry perturbation tests.  The model spin-up period is excluded.}
\label{fig:windSol_TseriesDryRg}
\end{figure}

\begin{figure}[here]
\includegraphics[width=1\textwidth]{img/solano_wind_drybkd_wetrg_d02_level0.png}
\caption{Time series of wind speed magnitude at 60 m AGL for the d02 grid point nearest the Solano wind farm, for the dry background and wet perturbation tests.  The model spin-up period is excluded.}
\label{fig:windSol_TseriesWetRg}
\end{figure}

In both the control and test cases, winds at Solano have a strong diurnal cycle (Figure \ref{fig:windSol_TseriesDryRg}(a)) and tend to be strongest on days with stronger synoptic forcing (July XX).  Solano wind speeds are greatest XX to XX hours and are weakest XX to XX hours (Figure \ref{fig:windSol_DiffDiurnalRg}(a)). Solano winds are strongest in lowest 300 m, particularly in hours XX-XX  (Figure \ref{fig:windSol_VertProfileDryRg}(a)).  The two-week average $u$-component of Solano winds ($u_{avg}$) is westerly at all heights and times, but at heights 300-1000 m the $v_{avg}$-component of the winds oscillates between northerly in night and morning to slightly southerly in afternoon and evening; below 300 m, $v_{avg}$ is southerly at all hours.

\begin{figure}[here]
\begin{subfigure}{0.6\textwidth}
\includegraphics[width=1\textwidth]{img/solano_controlwind_minusmean_CA0pt25_d02_level0.png}
\caption{}
\end{subfigure}
\begin{subfigure}{0.6\textwidth}
\includegraphics[width=1\textwidth]{img/solano_diurnalwind_dry_regions_d02_level0.png}
\caption{}
\end{subfigure}
\begin{subfigure}{0.6\textwidth}
\includegraphics[width=1\textwidth]{img/solano_diurnalwind_wet_regions_d02_level0.png}
\caption{}
\end{subfigure}
\caption{(a) Overlaid diurnal cycles of wind speed magnitude minus daily mean wind speed, at 60 m AGL for the d02 grid point nearest the Solano wind farm, for the CA-0.25 control case.  (b) and (c) Diurnally averaged differences in wind speed, at 60 m AGL for the d02 grid point nearest the Solano wind farm, for (b) the wet background/dry perturbation cases and (c) the dry background/wet perturbation cases.  Shading represents one standard deviation.}
\label{fig:windSol_DiffDiurnalRg}
\end{figure}

\begin{figure}[here]
\begin{subfigure}{0.49\textwidth}
\includegraphics[width=\textwidth]{img/windprof_hr_avg_CA0pt25.png}
\caption{}
\end{subfigure}
\begin{subfigure}{0.49\textwidth}
\includegraphics[width=\textwidth]{img/windprof_diff_hr_avg_dryCR.png}
\caption{}
\end{subfigure}
\begin{subfigure}{0.49\textwidth}
\includegraphics[width=\textwidth]{img/windprof_diff_hr_avg_dryCV.png}
\caption{}
\end{subfigure}
\begin{subfigure}{0.49\textwidth}
\includegraphics[width=\textwidth]{img/windprof_diff_hr_avg_drySN.png}
\caption{}
\end{subfigure}
\caption{Vertical profiles of $u$, $v$, and $|u|$, averaged by time of day over the whole simulation (colorbar indicates hour of day in local time), for (a) CA-0.25 control run, and averaged differences from control run for the dry regional test cases: (b) dryCR, (c) dryCV, (d) drySN.}
\label{fig:windSol_VertProfileDryRg}
\end{figure}

Regional low-level winds in the control case are consistent with previous studies of California surface winds (Figure \ref{fig:windSol_WindMapsRg} first column, \textbf{CITATIONS}); flow is strong through the Solano pass and splits into northward and southward branches in the Central Valley.  The southward branch is stronger than the northward branch at 06:00 and 14:00, and the northward branch strengthens at 18:00 and 00:00.  The pressure gradient from the San Francisco Bay to the Central Valley remains strong from 14:00 to 00:00 and weakens by 06:00.

\begin{figure}[here]
\includegraphics[angle=90,origin=c,width=1\textwidth]{img/wind_map_regions2.eps}
\caption{Wind speed (color shading) and direction (vectors) and pressure (color contours) at 110 m ASL on the d02 domain, averaged by hour over the two-week runs.  (a)-(d) CA-0.25 control case.  (e)-(h) dryCR case, changes in wind and pressure.  (i)-(l) dryCV case, changes in wind and pressure.  (m)-(p) drySN case, changes in wind and pressure.  Top row: average for hour 06:00 for the whole run; second row: average for hour 14:00; third row: average for hour 18:00; bottom row: average for hour 00:00.}
\label{fig:windSol_WindMapsRg}
\end{figure}

The Solano wind in the regional test cases differs from the control case by as much as 2.5-3 m/s, and the changes are larger on days when the synoptic wind forcing is weaker (Figures \ref{fig:windSol_TseriesDryRg} and \ref{fig:windSol_TseriesWetRg}, e.g. XX days). The largest changes occur when Central Valley soil moisture is perturbed (dryCR and wetCR).  

Drier soils on a wet background increase Solano winds in the afternoon and evening in all regional test cases (Figure \ref{fig:windSol_DiffDiurnalRg}(b)); the magnitude of this increase is largest in the dryCV case.  The increases in both the dryCV and drySN cases are greatest between 11:00 and 18:00 (1-2 m/s for dryCV and 0.5-1 m/s for drySN), while the increase in the dryCR case happens later, between 16:00 and 21:00 (0.5-1 m/s).  Importantly, such increases in the afternoon cause the daily wind speed ramp-up to shift earlier, because the increases occur at the same time as the control ramp-up.  Because there are no corresponding decreases at the hour of ramp-down, this means that the duration of the high-wind period also increases.  Predicting the timing of ramps is important for utilities, and these results suggest that soil moisture influences ramp-up timing at Solano.  Also, drier soils (especially in the Central Valley) increase the minimum wind (XX hours) on many individual days, especially days with weaker synoptic wind forcing (Figure \ref{fig:windSol_TseriesDryRg}), but these changes do not appear in the two-week average diurnal cycle (Figure \ref{fig:windSol_DiffDiurnalRg}(b)).

Wetter soils on a dry background have a smaller impact on Solano winds than drier soils on a wet background (Figure \ref{fig:windSol_DiffDiurnalRg}(c)).  Again, the decrease is largest in when Central Valley soil moisture is perturbed.  Both the wetCV and wetSN cases have wind decreases during 10:00-18:00 that are more than one standard deviation from zero (0.25-1.5 m/s for wetCV and 0.2-0.7 m/s for wetSN); for the wetCR case, there are no times of day with wind changes more than one standard deviation from zero.

\textit{Vertical profiles: vertical center of increases in each case.  Increases in u at all times and at all heights below 1700 m, but differences in magnitude, especially in morning.  Changes in v tend to amplify existing diurnal pattern, with more northerly in night and morning and more southerly in afternoon.  Degree to which the elevated increases propagate to the ground?}

\textit{All three test cases show an increase in $u$ at middle levels (500 m - 1000 m).  The increase is centered higher in the dryCR and drySN cases (around 1000 m) than in the dryCV case (around 700 m), and the increase extends to the near-surface winds to a greater degree in the dryCV case than in the dryCR or drySN case.}

The Solano wind changes occur in the context of larger regional wind and pressure changes (Figure \ref{fig:windSol_WindMapsRg}).  In all dry regional tests, the changes in winds and pressure gradients throughout the Central Valley are largest in the afternoon (second and third rows in Figure \ref{fig:windSol_WindMapsRg}).  In the dry Coast Range test (Figure \ref{fig:windSol_WindMapsRg} (e)-(h)), both wind and pressure changes are largest in the northern Central Valley, where cyclonic flow develops around the northern Coast Range.  In the dry Central Valley test (Figure \ref{fig:windSol_WindMapsRg} (i)-(l)), the pressure gradient from the San Francisco Bay to the Central Valley increases dramatically at 14:00, and these increases persist at 18:00 and to a lesser degree at 00:00. However, at 18:00, the zone of steepest pressure change has been pushed further eastward, and the strongest wind increases track this band of largest pressure gradient.  The pattern of wind increases at 00:00 is disorganized, and wind speeds decrease in the southern Central Valley at 06:00.  In the dry Sierra Nevada case (Figure \ref{fig:windSol_WindMapsRg} (m)-(p)), the pressure gradient strengthens moderately at 14:00 and 18:00, but pressure changes are minimal by 00:00 and 06:00.  Wind speeds increase through the Solano pass and the middle Central Valley in the afternoon (14:00 and 18:00), and by 18:00, the bands of largest wind increases have moved outward along Central Valley, again following zones of greatest pressure gradient increase.  Wind changes at night (00:00 and 06:00) are small and disorganized.

In summary, the Central Valley soil moisture influences the Solano turbine-level winds more strongly than does the Coast Range or Sierra Nevada soil moisture.  Drier soils in all regions, but especially in the Central Valley, increase Solano wind speeds during ramp-up and peak hours (afternoon and evening); drier soils in the Central Valley, especially, shift the daily wind ramp-up earlier.  Wetter soils, on the other hand, have a smaller effect on Solano winds; wetter Central Valley soils cause Solano winds to decrease during ramp-up and peak times (afternoon and evening.)

\subsubsection{Scaling of wind changes with Central Valley soil moisture}

\begin{figure}[here]
\includegraphics[width=1\textwidth]{img/solano_wind_CV0pt2_d02_level0.png}
\caption{Time series of wind speed magnitude at 60 m AGL for the d02 grid point nearest the Solano wind farm, for a range of Central Valley soil moisture values, with soil moisture = 0.2 in the Coast Range and Sierra Nevada.  (a) Wind speed time series, (b) time series of differences between test cases and control (CA-0.2).  The model spin-up period is excluded.}
\label{fig:windSol_TseriesWindCV}
\end{figure}

\begin{figure}[here]
\begin{subfigure}{0.6\textwidth}
\includegraphics[width=1\textwidth]{img/solano_controlwind_minusmean_CA0pt2_d02_level0.png}
\caption{}
\end{subfigure}
\begin{subfigure}{0.6\textwidth}
\includegraphics[width=1\textwidth]{img/solano_diurnalwind_CV_0pt2_d02_level0.png}
\caption{}
\end{subfigure}
\caption{(a) Overlaid diurnal cycles of wind speed magnitude minus daily mean wind speed, at 60 m AGL for the d02 grid point nearest the Solano wind farm, for the CA-0.2 control case.  (b) Diurnally averaged differences in wind speed, at 60 m AGL for the d02 grid point nearest the Solano wind farm, for a range of Central Valley soil moisture values.  Shading represents one standard deviation.}
\label{fig:windSol_DiffDiurnalCV0pt2}
\end{figure}

Drier Central Valley soils increase Solano winds, while wetter Central Valley soils decrease Solano winds (Figure \ref{fig:windSol_TseriesWindCV}.)  These changes can be up to 3 m/s, and they occur most consistently in the hours of 12:00 to 22:00 (Figure \ref{fig:windSol_DiffDiurnalCV0pt2}(b)), when they are 0.8-1.7 m/s on average in the driest Central Valley case (CV0.05).  The average changes during hours 12:00 to 22:00 scale nonlinearly with CV soil moisture (\textbf{shift stats figure}), with larger increases in wind per unit soil moisture decrease when the soil is dry.  The greater sensitivity to soil moisture when soils are drier is likely due to the greater sensitivity of surface heating to soil moisture when soils are drier (\textbf{forcing figure}), which is related to rapid declines in soil hydraulic conductivity and plant stomatal conductance in the moderate-to-dry soil moisture range (\textbf{cite something � Bonan? Sap flow paper? Transitional soil moisture regime paper of some kind?}).

\begin{figure}[here]
\includegraphics[width=1\textwidth]{img/shifts_CVsmois_d02.png}
\caption{Shifts in wind characteristics as a function of Central Valley soil moisture.  (a) Change in average wind speed during the maximum period (XX - XX hrs): symbol is the mean of the changes in average maximum wind speed over all days, and error bars are one standard deviation of the changes in average maximum wind speed.  (b) Change in average wind speed during the minimum period (XX - XX hrs); symbol and error bars as in (a).  (c) Change in hour of ramp-up, defined in the text; symbol and error bars as in (a).  (d) Change in hour of ramp-down, defined in the text; symbol and error bars as in (a).  \textbf{Might need to put equations in the text to clarify how these metrics are calculated.}}
\label{fig:windSol_WindShiftsCV}
\end{figure}

\subsubsection{Scaling of wind changes with area modified (??)}

\subsection{Physical mechanism}
\label{subsec:PhysMech}

Next we investigate the physical mechanism by which changes in soil moisture, especially in the Central Valley, influence near-surface winds at Solano.  

\subsubsection{Scaling analysis}

A scaling analysis of the terms of the momentum equation illustrates the relative importance of the momentum advection and pressure gradient terms in determining accelerations in westerly wind velocity $u$.  

\begin{equation}
\frac{\partial u}{\partial t} = -u\frac{\partial u}{\partial x} -\frac{1}{\rho} \frac{\partial p}{\partial x} - \kappa u
\end{equation}

We neglect the Coriolis term and the $y$-direction advection and pressure gradient because topographic channeling enforces nearly westerly flow in the Solano pass, and we treat friction as a linear function of wind speed.  We take 10 m/s as a typical value for $U$, 100 km as the length scale $L$ across the Solano pass from the San Francisco Bay to the Central Valley, +1 m/s as the typical $\Delta U$ from the San Francisco Bay to the Central Valley, -1 hPa as the typical pressure difference $\Delta p$ from the San Francisco Bay to the Central Valley, 1 kg/m$^3$ as the density $\rho$, and 7$\times$10$^{-5}$ s$^{-1}$ as the linear friction coefficient $\kappa$ [Zhong et al. 2004]. Then the advection term $\frac{-U^2}{L}$ is $\mathcal{O}$(-10$^{-3}$ m/s$^2$), the pressure gradient term $-\frac{1}{\rho} \frac{\Delta p}{L}$ is $\mathcal{O}$(10$^{-3}$ m/s$^2$), and the friction term $-\kappa U$ is $\mathcal{O}$(-10$^{-4}$ m/s$^2$).  Thus, the pressure gradient and momentum advection terms dominate the accelerations in $u$, and they work in opposite directions, with the pressure gradient tending to accelerate $u$ and the momentum advection tending to decelerate $u$.  During periods of greatest modeled wind acceleration or deceleration, the modeled magnitude of $\frac{\partial u}{\partial t}$ is \textbf{approximately 5$\times$10$^{-4}$ to 10$^{-3}$ m/s$^2$ WHICH IS IT? ($\Delta U$ = 4 m/s, $\Delta t$ = 2 hr)}.  Thus, the balance of the pressure gradient and momentum advection terms can account for the modeled accelerations and decelerations.

\subsubsection{Pressure patterns driving wind}

\begin{figure}[here]
\includegraphics[width=1\textwidth]{img/corr_wind_panom_lev200_lag3.png}
\caption{Regression between 60 m AGL Solano wind speed (purple diamond) and the horizontal pressure anomaly, with wind speed lagging pressure gradient by 90 min, for the CA-0.25 run, domain d02.  Left column: linear regression slope; right column: Pearson's r correlation coefficient for the linear regression; top row: regression using $u$ component of the Solano wind; bottom row: regression using $v$ component of the Solano wind.}
\label{fig:windSol_corrPanomUmap}
\end{figure}

The scaling analysis shows that the pressure gradient plays an important role in driving accelerations in $u$ in the Solano pass.  As such, we seek to identify the vertical level and horizontal distance at which the pressure gradient drives the winds, and we investigate how changes in the land surface energy balance affect pressure at this level and horizontal locations.  First, we linearly regress turbine-height $u$ (60 m AGL) against the horizontal pressure anomalies at each grid point and at each height, as described in Section XX.  The regression slopes and Pearson's r correlation coefficient for each grid point using pressure anomalies from 200 m AGL are shown in Figure \ref{fig:windSol_corrPanomUmap}; results for pressure anomalies from other heights are not shown because the patterns are very similar for all heights below 330 m, and the correlations are much weaker for heights above 330 m.  The correlations peak when the wind time series lags the pressure anomaly time series by 90 min, and this lag was used to create Figure \ref{fig:windSol_corrPanomUmap}.  Figure \ref{fig:windSol_corrPanomUmap} shows results for the CA-0.25 run; results for other runs were very similar and are not shown.  

Winds at Solano tend to be higher when the blue regions in the left column of Figure \ref{fig:windSol_corrPanomUmap} (northern and southern Central Valley) have lower pressure than the rest of the domain, and when the red regions (San Francisco Bay and central coast) have higher pressure than the rest of the domain.  Darker colors in those same regions indicate tight positive (red) or negative (blue) correlations between the Solano wind time series and the pressure anomaly time series in those regions.  

We repeat the regression analysis to find regions with pressure changes in the test runs that correlate with changes in the Solano wind.  The slopes and correlation coefficients of the regression between test-minus-control changes in pressure anomalies and test-minus-control changes in Solano wind are shown in Figures \ref{fig:windSol_corrDiffPanomUmapDryRg} (dry perturbation regions with wet background) and \ref{fig:windSol_corrDiffPanomUmapWetRg} (wet perturbation regions with dry background).  Decreases in pressure in XX regions are strongly related to increases in Solano wind speed in XX test cases.  The relationship is weaker in these regions in XX test cases.  \textit{Say anything about difference in patterns for u and v?}  \textbf{Draw boxes on figures, and say that Solano wind is deemed to be most sensitive to pressure changes in these regions, and these boxes will be used in the following analysis.}

\begin{figure}[here]
\begin{subfigure}{0.6\textwidth}
\includegraphics[width=1\textwidth]{img/corr_dwind_dpanom_lev200_lag3_dryCR.png}
\caption{}
\end{subfigure}
\begin{subfigure}{0.6\textwidth}
\includegraphics[width=1\textwidth]{img/corr_dwind_dpanom_lev200_lag3_dryCV.png}
\caption{}
\end{subfigure}
\begin{subfigure}{0.6\textwidth}
\includegraphics[width=1\textwidth]{img/corr_dwind_dpanom_lev200_lag3_drySN.png}
\caption{}
\end{subfigure}
\caption{As in Figure \ref{fig:windSol_corrPanomUmap}, but linear regression performed with changes in wind and pressure anomalies from control case to test case.  (a) dryCR case, (b) dryCV case, (c) drySN case.}
\label{fig:windSol_corrDiffPanomUmapDryRg}
\end{figure}

\begin{figure}[here]
\begin{subfigure}{0.6\textwidth}
\includegraphics[width=1\textwidth]{img/corr_dwind_dpanom_lev200_lag3_wetCR.png}
\caption{}
\end{subfigure}
\begin{subfigure}{0.6\textwidth}
\includegraphics[width=1\textwidth]{img/corr_dwind_dpanom_lev200_lag3_wetCV.png}
\caption{}
\end{subfigure}
\begin{subfigure}{0.6\textwidth}
\includegraphics[width=1\textwidth]{img/corr_dwind_dpanom_lev200_lag3_wetSN.png}
\caption{}
\end{subfigure}
\caption{As in Figure \ref{fig:windSol_corrPanomUmap}, but linear regression performed with changes in wind and pressure anomalies from control case to test case.  (a) wetCR case, (b) wetCV case, (c) wetSN case.}
\label{fig:windSol_corrDiffPanomUmapWetRg}
\end{figure}

The boxes in Figure \ref{fig:windSol_corrPanomUmap} \textbf{(need to add these boxes to the figure)} outline regions with both large slopes and large r-values, implying that Solano winds are highly sensitive to the pressure anomalies.  When pressure in box A is higher than the rest of the domain, 60 m AGL winds at Solano are faster (positive slopes and r-values); in contrast, when pressure in box B is lower than the rest of the domain, 60 m AGL winds at Solano are faster (negative slopes and r-values.)

Winds at Solano lag the pressure gradient between box A and box B (using the average pressure for each box at a given level), with correlation peaking at a lag of XX hours (Figure \ref{fig:windSol_lagcorrPgradU}).  The sensitivity (as measured by the linear regression slope) and the correlation are maximum for the 250 m ASL pressure gradient, although the correlation and sensitivity are strong for pressure gradients at all levels below 350 m ASL.  Figure \ref{fig:windSol_lagcorrPgradU} shows results for the CA-0.2 run only; results for other runs are similar, although the peak lag varies from XX in the XX run to XX in the XX run.  \textbf{Discuss sensitivity to higher (550 m) p-grad, but at a longer lag and with poorer correlation.  What does this mean?}

\begin{figure}[here]
\includegraphics[width=1\textwidth]{img/lag_corr_p_u_CA0pt2.png}
\caption{Lag correlation between the pressure gradient from box A to boxB (leading) and Solano wind speed (lagging), for the CA-0.2 run, domain d02.  (a) Linear regression slope between wind speed pressure gradient as a function of lag hours, for different vertical levels of the pressure gradient; (b) Pearson's r correlation coefficient as a function of lag hours, for different vertical levels of the pressure gradient.  \textbf{Need to remove the left two panels, only show d02.  Also need to remove the second level wind lines.}}
\label{fig:windSol_lagcorrPgradU}
\end{figure}

\subsubsection{Local heating and advective controls on pressure gradient}

\textbf{Discuss how temperature changes are known to affect pressure (summarizing arguments from sea breeze review paper.  This may need to be moved to intro or discussion, and could reference it here.)}

Air temperature at XX m depends on air temperature 2 m above the surface; the dependence is linear during the day, but at night, the air at 250 m stays warmer than the air at 2 m, as expected from radiative cooling of the land surface and resulting stabilization and suppressed mixing of the lower atmosphere.  Air temperature at XX m in box B is sensitive to near-surface air temperature in XX regions, as measured by linear regression slopes and r-values (Figure XX).  Temperature at 2 m, in turn, depends strongly on surface skin temperature, with large linear regression slopes and r-values (Figure XX).  Finally, surface skin temperature depends greatly on the soil-moisture-dependent partitioning of land surface energy fluxes between evapotranspiration and sensible heat; Figure XX shows that the drier XX case has much higher surface skin temperature at midday in the Central Valley than does the wetter XX case.

\subsubsection{Summary of mechanism}

Figure XX summarizes the changes in land surface heating, air temperature, and pressure that lead to the changes in Solano winds, for XX cases.  In the drier cases, surface temperature (increases earlier?) and reaches midday peaks that are XX to XX deg C higher than the control (or wetter) cases.  This leads to 2 m air temperatures that are approximately XX deg C higher (and any shift in timing?), and to XX m air temperatures that are XX deg C higher (and any shift in timing?).  Pressure at box A decreases by XX and at box B decreases by XX, so that the pressure gradient (is lower during the day?  shifts in timing?).  As a result, winds at Solano, which lag pressure gradient by approximately XX hrs, increase by XX m/s at XX times, and the ramp-up and ramp-down periods (both shift earlier, following the pressure gradient forcing.)

\textbf{Still to flesh out: Investigation of controls on air temperature at the important level and the important horizontal area, using energy balance.  Times when advection vs surface heating vs entrainment matter more.  How sensitive each of these components are to the change in soil moisture.}


%\end{document}