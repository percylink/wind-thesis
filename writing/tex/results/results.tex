
%\documentclass[12pt]{amsart}
%\usepackage{geometry} % see geometry.pdf on how to lay out the page. There's lots.
%\usepackage{datetime}
%\usepackage{setspace}
%\doublespacing
%\geometry{a4paper} % or letter or a5paper or ... etc
%% \geometry{landscape} % rotated page geometry
%
%% See the ``Article customise'' template for come common customisations
%
%\title{Wind chapter}
%\author{Percy Link}
%\date{\currenttime \ \today} % delete this line to display the current date
%
%%%% BEGIN DOCUMENT
%\begin{document}
%
%\maketitle

\section{Results}

We first characterize the differences in Solano turbine-level wind resulting from the soil moisture tests (Section \ref{subsec:CharWindChanges}).  We then investigate the physical mechanism linking changes in soil moisture to changes in Solano wind timing and magnitude (Section \ref{subsec:PhysMech}).

\subsection{Characterization of Solano wind sensitivity to soil moisture}
\label{subsec:CharWindChanges}

\subsubsection{Regional sensitivity}

Before presenting the Solano winds, we quantify the synoptic and land-surface-heating forcing in each case, and we characterize the control case winds.  The synoptic forcing, measured by wind speed at XX m, varies little between the model runs (Figure XX), and synoptic winds are weak on XX days and strong on XX days.  Soil moisture declines by XX m$^3$/m$^3$ or less before being reset each midnight (Figure XX).  Land surface heating \textbf{DESCRIBE DIFFERENCES HERE}

\begin{figure}[here]
\includegraphics[width=1\textwidth]{img/solano_wind_wetbkd_dryrg_d02_level0.png}
\caption{Time series of wind speed magnitude at 60 m AGL for the d02 grid point nearest the Solano wind farm, for the wet background and dry perturbation tests.  The model spin-up period is excluded.}
\label{fig:windSol_TseriesDryRg}
\end{figure}

\begin{figure}[here]
\includegraphics[width=1\textwidth]{img/solano_wind_drybkd_wetrg_d02_level0.png}
\caption{Time series of wind speed magnitude at 60 m AGL for the d02 grid point nearest the Solano wind farm, for the dry background and wet perturbation tests.  The model spin-up period is excluded.}
\label{fig:windSol_TseriesWetRg}
\end{figure}

In both the control and test cases, winds at Solano have a strong diurnal cycle (Figure \ref{fig:windSol_TseriesDryRg}(a)) and tend to be strongest on days with stronger synoptic forcing (July XX).  Solano wind speeds are greatest XX to XX hours and are weakest XX to XX hours (Figure \ref{fig:windSol_DiffDiurnalDryRg}(a) and Figure \ref{fig:windSol_DiffDiurnalWetRg}(a)). Solano winds are strongest in lowest 300 m, particularly in hours XX-XX  (Figure \ref{fig:windSol_VertProfileDryRg}(a)).  The two-week average $u$-component of Solano winds ($u_{avg}$) is westerly at all heights and times, but at heights 300-1000 m the $v_{avg}$-component of the winds oscillates between northerly in night and morning to slightly southerly in afternoon and evening; below 300 m, $v_{avg}$ is southerly at all hours.

\begin{figure}[here]
\begin{subfigure}{0.5\textwidth}
\includegraphics[width=1\textwidth]{img/solano_controlwind_minusmean_CA0pt25_d02_level0.png}
\caption{}
\end{subfigure}
\begin{subfigure}{0.5\textwidth}
\includegraphics[width=1\textwidth]{img/solano_diurnalwind_dry_regions_d02_level0.png}
\caption{}
\end{subfigure}
\caption{(a) Overlaid diurnal cycles of wind speed magnitude minus daily mean wind speed, at 60 m AGL for the d02 grid point nearest the Solano wind farm, for the CA-0.25 control case.  (b) Diurnally averaged differences in wind speed, at 60 m AGL for the d02 grid point nearest the Solano wind farm, for the wet background/dry perturbation cases.  Shading represents one standard deviation.}
\label{fig:windSol_DiffDiurnalDryRg}
\end{figure}

\begin{figure}[here]
\begin{subfigure}{0.5\textwidth}
\includegraphics[width=1\textwidth]{img/solano_controlwind_minusmean_CA0pt1_d02_level0.png}
\caption{}
\end{subfigure}
\begin{subfigure}{0.5\textwidth}
\includegraphics[width=1\textwidth]{img/solano_diurnalwind_wet_regions_d02_level0.png}
\caption{}
\end{subfigure}
\caption{(a) Overlaid diurnal cycles of wind speed magnitude minus daily mean wind speed, at 60 m AGL for the d02 grid point nearest the Solano wind farm, for the CA-0.1 control case.  (b) Diurnally averaged differences in wind speed, at 60 m AGL for the d02 grid point nearest the Solano wind farm, for the dry background/wet perturbation cases.  Shading represents one standard deviation.}
\label{fig:windSol_DiffDiurnalWetRg}
\end{figure}

\begin{figure}[here]
\begin{subfigure}{0.49\textwidth}
\includegraphics[width=\textwidth]{img/windprof_hr_avg_CA0pt25.png}
\caption{}
\end{subfigure}
\begin{subfigure}{0.49\textwidth}
\includegraphics[width=\textwidth]{img/windprof_diff_hr_avg_dryCR.png}
\caption{}
\end{subfigure}
\begin{subfigure}{0.49\textwidth}
\includegraphics[width=\textwidth]{img/windprof_diff_hr_avg_dryCV.png}
\caption{}
\end{subfigure}
\begin{subfigure}{0.49\textwidth}
\includegraphics[width=\textwidth]{img/windprof_diff_hr_avg_drySN.png}
\caption{}
\end{subfigure}
\caption{Vertical profiles of $u$, $v$, and $|u|$, averaged by time of day over the whole simulation (colorbar indicates hour of day in local time), for (a) CA-0.25 control run, and averaged differences from control run for the dry regional test cases: (b) dryCR, (c) dryCV, (d) drySN.}
\label{fig:windSol_VertProfileDryRg}
\end{figure}

Regional low-level winds in the control case are consistent with previous studies of California surface winds (Figure \ref{fig:windSol_WindMapsRg} first column, \textbf{CITATIONS}); flow is strong through the Solano pass and splits into northward and southward branches in the Central Valley.  The southward branch is stronger than the northward branch at 06:00 and 14:00, and the northward branch strengthens at 18:00 and 00:00.  The pressure gradient from the San Francisco Bay to the Central Valley remains strong from 14:00 to 00:00 and weakens by 06:00.

\begin{figure}[here]
\includegraphics[angle=90,origin=c,width=1\textwidth]{img/wind_map_regions2.eps}
\caption{Wind speed (color shading) and direction (vectors) and pressure (color contours) at 110 m ASL on the d02 domain, averaged by hour over the two-week runs.  (a)-(d) CA-0.25 control case.  (e)-(h) dryCR case, changes in wind and pressure.  (i)-(l) dryCV case, changes in wind and pressure.  (m)-(p) drySN case, changes in wind and pressure.  Top row: average for hour 06:00 for the whole run; second row: average for hour 14:00; third row: average for hour 18:00; bottom row: average for hour 00:00.}
\label{fig:windSol_WindMapsRg}
\end{figure}

The Solano wind in the regional test cases differs from the control case by as much as 2.5-3 m/s, and the changes are larger on days when the synoptic wind forcing is weaker (Figures \ref{fig:windSol_TseriesDryRg} and \ref{fig:windSol_TseriesWetRg}, e.g. XX days). The largest changes occur when Central Valley soil moisture is perturbed (dryCR and wetCR).  

Drier soils on a wet background increase Solano winds in the afternoon and evening in all regional test cases (Figure \ref{fig:windSol_DiffDiurnalDryRg}(b)); the magnitude of this increase is largest in the dryCV case.  The increases in both the dryCV and drySN cases are greatest between 11:00 and 18:00 (1-2 m/s for dryCV and 0.5-1 m/s for drySN), while the increase in the dryCR case happens later, between 16:00 and 21:00 (0.5-1 m/s).  Importantly, such increases in the afternoon cause the daily wind speed ramp-up to shift earlier, because the increases occur at the same time as the control ramp-up.  Because there are no corresponding decreases at the hour of ramp-down, this means that the duration of the high-wind period also increases.  Predicting the timing of ramps is important for utilities, and these results suggest that soil moisture influences ramp-up timing at Solano.  Also, drier soils (especially in the Central Valley) increase the minimum wind (XX hours) on many individual days, especially days with weaker synoptic wind forcing (Figure \ref{fig:windSol_TseriesDryRg}), but these changes do not appear in the two-week average diurnal cycle (Figure \ref{fig:windSol_DiffDiurnalDryRg}(b)).

Wetter soils on a dry background have a smaller impact on Solano winds than drier soils on a wet background (Figure \ref{fig:windSol_DiffDiurnalWetRg}(b)).  Again, the decrease is largest in when Central Valley soil moisture is perturbed.  Both the wetCV and wetSN cases have wind decreases during 10:00-18:00 that are more than one standard deviation from zero (0.25-1.5 m/s for wetCV and 0.2-0.7 m/s for wetSN); for the wetCR case, there are no times of day with wind changes more than one standard deviation from zero.

\textit{Vertical profiles: vertical center of increases in each case.  Increases in u at all times and at all heights below 1700 m, but differences in magnitude, especially in morning.  Changes in v tend to amplify existing diurnal pattern, with more northerly in night and morning and more southerly in afternoon.  Degree to which the elevated increases propagate to the ground?}

\textit{All three test cases show an increase in $u$ at middle levels (500 m - 1000 m).  The increase is centered higher in the dryCR and drySN cases (around 1000 m) than in the dryCV case (around 700 m), and the increase extends to the near-surface winds to a greater degree in the dryCV case than in the dryCR or drySN case.}

The Solano wind changes occur in the context of larger regional wind and pressure changes (Figure \ref{fig:windSol_WindMapsRg}).  In all dry regional tests, the changes in winds and pressure gradients throughout the Central Valley are largest in the afternoon (second and third rows in Figure \ref{fig:windSol_WindMapsRg}).  In the dry Coast Range test (Figure \ref{fig:windSol_WindMapsRg} (e)-(h)), both wind and pressure changes are largest in the northern Central Valley, where cyclonic flow develops around the northern Coast Range.  In the dry Central Valley test (Figure \ref{fig:windSol_WindMapsRg} (i)-(l)), the pressure gradient from the San Francisco Bay to the Central Valley increases dramatically at 14:00, and these increases persist at 18:00 and to a lesser degree at 00:00. However, at 18:00, the zone of steepest pressure change has been pushed further eastward, and the strongest wind increases track this band of largest pressure gradient.  The pattern of wind increases at 00:00 is disorganized, and wind speeds decrease in the southern Central Valley at 06:00.  In the dry Sierra Nevada case (Figure \ref{fig:windSol_WindMapsRg} (m)-(p)), the pressure gradient strengthens moderately at 14:00 and 18:00, but pressure changes are minimal by 00:00 and 06:00.  Wind speeds increase through the Solano pass and the middle Central Valley in the afternoon (14:00 and 18:00), and by 18:00, the bands of largest wind increases have moved outward along Central Valley, again following zones of greatest pressure gradient increase.  Wind changes at night (00:00 and 06:00) are small and disorganized.

In summary, the Central Valley soil moisture influences the Solano turbine-level winds more strongly than does the Coast Range or Sierra Nevada soil moisture.  Drier soils in all regions, but especially in the Central Valley, increase Solano wind speeds during ramp-up and peak hours (afternoon and evening); drier soils in the Central Valley, especially, shift the daily wind ramp-up earlier.  Wetter soils, on the other hand, have a smaller effect on Solano winds; wetter Central Valley soils cause Solano winds to decrease during ramp-up and peak times (afternoon and evening.)

\subsubsection{Scaling of wind changes with Central Valley soil moisture}

\begin{figure}[here]
\includegraphics[width=1\textwidth]{img/solano_wind_CV0pt2_d02_level0.png}
\caption{Time series of wind speed magnitude at 60 m AGL for the d02 grid point nearest the Solano wind farm, for a range of Central Valley soil moisture values, with soil moisture = 0.2 in the Coast Range and Sierra Nevada.  (a) Wind speed time series, (b) time series of differences between test cases and control (CA-0.2).  The model spin-up period is excluded.}
\label{fig:windSol_TseriesWindCV}
\end{figure}

\begin{figure}[here]
\begin{subfigure}{0.6\textwidth}
\includegraphics[width=1\textwidth]{img/solano_controlwind_minusmean_CA0pt2_d02_level0.png}
\caption{}
\end{subfigure}
\begin{subfigure}{0.6\textwidth}
\includegraphics[width=1\textwidth]{img/solano_diurnalwind_CV_0pt2_d02_level0.png}
\caption{}
\end{subfigure}
\caption{(a) Overlaid diurnal cycles of wind speed magnitude minus daily mean wind speed, at 60 m AGL for the d02 grid point nearest the Solano wind farm, for the CA-0.2 control case.  (b) Diurnally averaged differences in wind speed, at 60 m AGL for the d02 grid point nearest the Solano wind farm, for a range of Central Valley soil moisture values.  Shading represents one standard deviation.}
\label{fig:windSol_DiffDiurnalCV0pt2}
\end{figure}

Drier Central Valley soils increase Solano winds, while wetter Central Valley soils decrease Solano winds (Figure \ref{fig:windSol_TseriesWindCV}.)  These changes can be up to 3 m/s, and they occur most consistently in the hours of 12:00 to 22:00 (Figure \ref{fig:windSol_DiffDiurnalCV0pt2}(b)), when they are 0.8-1.7 m/s on average in the driest Central Valley case (CV0.05).  The average changes during hours 12:00 to 22:00 scale nonlinearly with CV soil moisture (\textbf{shift stats figure}), with larger increases in wind per unit soil moisture decrease when the soil is dry.  The greater sensitivity to soil moisture when soils are drier is likely due to the greater sensitivity of surface heating to soil moisture when soils are drier (\textbf{forcing figure}), which is related to rapid declines in soil hydraulic conductivity and plant stomatal conductance in the moderate-to-dry soil moisture range (\textbf{cite something � Bonan? Sap flow paper? Transitional soil moisture regime paper of some kind?}).

%\begin{figure}[here]
%\includegraphics[width=1\textwidth]{img/shifts_CVsmois_d02.png}
%\caption{Shifts in wind characteristics as a function of Central Valley soil moisture.  (a) Change in average wind speed during the maximum period (XX - XX hrs): symbol is the mean of the changes in average maximum wind speed over all days, and error bars are one standard deviation of the changes in average maximum wind speed.  (b) Change in average wind speed during the minimum period (XX - XX hrs); symbol and error bars as in (a).  (c) Change in hour of ramp-up, defined in the text; symbol and error bars as in (a).  (d) Change in hour of ramp-down, defined in the text; symbol and error bars as in (a).  \textbf{Might need to put equations in the text to clarify how these metrics are calculated.}}
%\label{fig:windSol_WindShiftsCV}
%\end{figure}

\subsection{Physical mechanism}
\label{subsec:PhysMech}

Next we investigate the physical mechanism by which changes in soil moisture, especially in the Central Valley, influence near-surface winds at Solano.  

The strong diurnal component of the Solano wind (Figures \ref{fig:windSol_TseriesDryRg} and \ref{fig:windSol_TseriesWetRg}) and the wind speed peak at low levels (Figure \ref{fig:windSol_VertProfileDryRg}) suggest significant forcing by land surface heating (consistent with Zhong \textit{et al.} [2004] and Mansbach [2010]).  In order to explore the influence of land surface forcing on terms in the momentum budget, we model the topographically channeled wind at Solano as simple one-dimensional flow through the pass governed by a one-dimensional momentum equation,

\begin{equation}
\frac{\partial u}{\partial t} = -u\frac{\partial u}{\partial x} -\frac{1}{\rho} \frac{\partial p}{\partial x} - F(\frac{\partial u}{\partial z}, N),
\end{equation}
neglecting Coriolis because of the topographic constraint on wind direction.  The first term on the right hand side is the advection of momentum, the second term is the pressure gradient force, and the third term is frictional dissipation as an unspecified function of vertical wind shear and stability (quantified by $N$, the Brunt-V\"ais\"al\"a frequency).  We are interested in the importance of each of these terms through the diurnal cycle and in the influence of soil moisture changes on each of these terms.

\subsubsection{Driving pressure gradient}

\begin{figure}[here]
\includegraphics[width=1\textwidth]{img/corr_wind_panom_lev110_lag12_CA0pt25.png}
\caption{CA-0.25 case: Solano 60 m AGL (110 m ASL) wind linearly regressed against horizontal pressure anomaly at 110 m ASL, with wind lagging pressure by 6 hours.  Top row: $u$-component of wind; bottom row: $v$-component of wind.  Left column: linear regression slope; right column: correlation coefficient.  Gray boxes outline areas used to calculate pressure gradients; ocean box is ``OCN", northern box in the Central valley is ``NCV", and southern box in Central Valley is ``SCV".}
\label{fig:windSol_CorrMap0pt25}
\end{figure}

\begin{figure}[here]
\includegraphics[width=1\textwidth]{img/corr_wind_panom_lev110_lag12_CA0pt1.png}
\caption{CA-0.1 case: Solano 60 m AGL (110 m ASL) wind linearly regressed against horizontal pressure anomaly at 110 m ASL, with wind lagging pressure by 6 hours.  Top row: $u$-component of wind; bottom row: $v$-component of wind.  Left column: linear regression slope; right column: correlation coefficient.}
\label{fig:windSol_CorrMap0pt1}
\end{figure}

In order to identify the pressure gradient most relevant to Solano winds, we linearly regress the horizontal pressure anomaly against Solano turbine-level wind (Section \ref{subsubsec:WindPresRegression}).  The regression is repeated for pressure at a range of heights and for a range of lag times, with wind lagging pressure.  The regression slopes and correlation coefficients for the control cases (CA-0.25, Figure \ref{fig:windSol_CorrMap0pt25}, and CA-0.1, Figure \ref{fig:windSol_CorrMap0pt1}) show a consistent pattern of positive slopes and correlation coefficients over the ocean near the central coast (meaning higher pressure in those regions corresponds to faster wind), and negative slopes and correlation coefficients throughout the Central Valley, especially just north and south of the Solano pass (meaning lower pressure in those regions corresponds to faster wind).

\begin{figure}[here]
\includegraphics[width=1\textwidth]{img/pgrad_wind_CA0pt1_level110.png}
\caption{CA-0.1 control case: Solano 60 m AGL (110 m ASL) wind (black) and average pressure difference OCN box minus NCV box (blue) and OCN box minus SCV box (green) at 110 m ASL.  The sign of the pressure difference is chosen to correspond with the $\frac{-1}{\rho} \frac{\partial p}{\partial x}$ term in the momentum budget.}
\label{fig:windSol_PgradWind}
\end{figure}

\begin{figure}[here]
\includegraphics[width=1\textwidth]{img/lag_corr_pdiff_combo_ncv_scv_d02_CA0pt1.png}
\caption{CA-0.1 control case: lagged linear regression of Solano 60 m AGL (110 m ASL) wind against average pressure difference NCV box minus OCN box (dashed) and SCV box minus OCN box (solid).  Correlations are only shown for speed, because u and v velocity component results are very similar.}
\label{fig:windSol_LagCorrCtrl}
\end{figure}

Thus, we choose three boxes, shown in gray in Figures \ref{fig:windSol_CorrMap0pt25} to \ref{fig:windSol_CorrMap0pt1}, to represent the pressure gradient driving Solano winds.  The lag of wind speed behind pressure gradient is evident in the time series of Solano wind and pressure difference between the boxes (Figure \ref{fig:windSol_PgradWind}).  Not only does the pressure difference peak 5-6 hours before the Solano wind speed, but the peak pressure difference rapidly dissipates (after 1-2 hours \textbf{CHECK THIS}), while the peak wind speed persists for 6-8 hours.  Using the pressure difference between the three boxes, we confirm that the peak sensitivity (as measured by the regression slope) and correlation occur with pressure at 110 m ASL (although the metrics are high at all levels below $\sim$300 m) with a lag of $\sim$4-5 hr for NCV-OCN pressure and $\sim$5-6.5 hr for SCV-OCN pressure (Figure \ref{fig:windSol_LagCorrCtrl}).

\begin{figure}[here]
\begin{subfigure}{0.6\textwidth}
\includegraphics[width=\textwidth]{img/corr_dwind_dpanom_lev110_lag2_dryCR.png}
\caption{}
\end{subfigure}
\begin{subfigure}{0.6\textwidth}
\includegraphics[width=\textwidth]{img/corr_dwind_dpanom_lev110_lag2_dryCV.png}
\caption{}
\end{subfigure}
\begin{subfigure}{0.6\textwidth}
\includegraphics[width=\textwidth]{img/corr_dwind_dpanom_lev110_lag2_drySN.png}
\caption{}
\end{subfigure}
\caption{Correlation between Solano wind changes at 60 m AGL and horizontal pressure anomaly of change in 110 m ASL pressure, with a lag of 60 min, for the dry regional test cases minus control: (a) dryCR, (b) dryCV, (c) drySN.  The order of panels is as in Figure \ref{fig:windSol_CorrMap0pt25}.}
\label{fig:windSol_CorrMapDryRg}
\end{figure}

\begin{figure}[here]
\begin{subfigure}{0.6\textwidth}
\includegraphics[width=\textwidth]{img/corr_dwind_dpanom_lev110_lag2_wetCR.png}
\caption{}
\end{subfigure}
\begin{subfigure}{0.6\textwidth}
\includegraphics[width=\textwidth]{img/corr_dwind_dpanom_lev110_lag2_wetCV.png}
\caption{}
\end{subfigure}
\begin{subfigure}{0.6\textwidth}
\includegraphics[width=\textwidth]{img/corr_dwind_dpanom_lev110_lag2_wetSN.png}
\caption{}
\end{subfigure}
\caption{Correlation between Solano wind changes at 60 m AGL and horizontal pressure anomaly of change in 110 m ASL pressure, with a lag of 60 min, for the wet regional test cases minus control: (a) wetCR, (b) wetCV, (c) wetSN.  The order of panels is as in Figure \ref{fig:windSol_CorrMap0pt25}.}
\label{fig:windSol_CorrMapWetRg}
\end{figure}

\begin{figure}[here]
\includegraphics[width=1\textwidth]{img/lag_corr_dpdiff_combo_ncv_scv_d02_testminusctrl.png}
\caption{Dry regional perturbation cases: lagged linear regression of test-minus-control Solano 60 m AGL (110 m ASL) wind against test-minus-control average pressure difference NCV box minus OCN box (dashed) and SCV box minus OCN box (solid).  Top row: dryCR; middle row: dryCV; bottom row: drySN.}
\label{fig:windSol_LagCorrTest}
\end{figure}

Repeating the regression analysis using test-minus-control changes in pressure and Solano wind, a similar spatial pattern of sensitivity emerges (positive correlation between Solano wind changes and pressure changes in the central coastal ocean, and negative correlation between Solano wind changes and pressure changes in the Central Valley), albeit with weaker correlation (Figures \ref{fig:windSol_CorrMapDryRg} and \ref{fig:windSol_CorrMapWetRg}).  Again, the changes in wind speed correlate maximally with the changes in pressure at the lowest altitudes ($\le$ 200 m, Figure \ref{fig:windSol_LagCorrTest}).  In contrast to the control cases which have a 5-6 hour lag, the test-minus-control changes correlate maximally at a lag of $\le$ 1 hour (Figure \ref{fig:windSol_LagCorrTest}).

\subsubsection{Temperature controls on pressure}

The diurnal variations in low-level pressure correspond to diurnal variations in boundary layer temperature, because pressure is the vertical integral of the weight of the air column above a point, and changes in temperature change the density and thus the weight of the air column.
\begin{itemize}
\item Diurnal temperature variations are larger in the Central Valley than over the ocean.  The temperature at low levels (XX altitudes) declines quickly after sunset, but the temperature aloft in the boundary layer (XX altitudes) remains warmer until XX hour.
\item Only show one time-height plot?  As long as SCV and NCV versions are very similar.
\item The difference in temperature peaks at XX time (same as peak of CV temperature, or different? later?).  The temperature difference at low levels drops after sunset (and even reverses sign, as the near-surface air in the Central Valley cools more quickly than that over the ocean?), but the temperature difference remains positive (XX degrees C) at XX altitudes.  The near-surface cooling in the Central Valley is due to a combination of radiative cooling of the land surface and advection of cooler marine air by the low-level wind maximum.
\item What pressure difference does: when does it rise relative to the rise of the temperature difference?  What is its magnitude when temperature is elevated aloft but cool at the surface?
\end{itemize}

The changes in soil moisture affect the pressure gradient by changing surface heat fluxes and thus air temperature.
\begin{itemize}
\item Temperature changes in each box are XX magnitude at XX times.  Changes in the temperature difference between the boxes are largest at XX times and are of XX magnitude, which is XX fraction of the temperature difference in the control case.
\item Temperature changes aloft persist for XX hours after sunset, but only for XX hours or not at all at the surface.  These changes are or are not reflected in the changes in temperature difference between boxes.
\item Changes in the pressure difference between boxes are largest at XX hour, which corresponds to the largest changes in temperature difference at XX level.  The persistence of elevated temperatures aloft in the Central Valley does XX to the pressure gradient.
\end{itemize}

\subsubsection{Terms of the momentum budget - importance of momentum advection and friction - MAKE THIS BETTER}

\textit{Estimating the momentum advection and pressure gradient terms in Equation XX from model output, shown in Figure XX, we would predict the du/dt shown in Figure XX, giving a Solano diurnal wind cycle with phase and amplitude like that in Figure XX.  This is different than the modeled wind diurnal cycle (diurnal cycle figures previously), and the discrepancy is due to the friction term.  The residual of du/dt minus momentum advection and pressure gradient force is shown in Figure XX; this is an estimate of the $F(u)$ term in Equation XX.  This residual has the expected diurnal shape of friction, with high values in the daytime when convective turbulence creates a uniform boundary layer and high momentum flux to the ground, and low values at night when the residual upper boundary layer decouples from the shallow stable nocturnal boundary layer and is buffered from friction with the surface.  Indeed, the diurnal course of modeled $u*$ at Solano has the same timing as the $F(\frac{\partial u}{\partial z}, N)$ term calculated from the residual (ustar figure).  The drop in friction at hours XX explains the sustained high winds at hours XX even after the pressure gradient force declines to values XX fraction of the peak daytime values.}

\textit{Changing the soil moisture changes the pressure gradient term.  In XX case, it changes the pressure gradient at XX hour by XX fraction.  These changes are related to air temperature changes at XX levels at XX hours.  Note differences between cases.  Note any persistence of temperature increases aloft after nightfall.}

\textit{The changes in the pressure gradient force (do or do not) occur at the same time as the changes in wind.  There are (or are not) changes in momentum advection and friction, (and they happen at XX hours.)}

\subsubsection{Summary of mechanism}



%\end{document}