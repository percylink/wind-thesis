
%\documentclass[12pt]{amsart}
%\usepackage{geometry} % see geometry.pdf on how to lay out the page. There's lots.
%\usepackage{datetime}
%\usepackage{setspace}
%\doublespacing
%\geometry{a4paper} % or letter or a5paper or ... etc
%% \geometry{landscape} % rotated page geometry
%
%% See the ``Article customise'' template for come common customisations
%
%\title{Wind chapter}
%\author{Percy Link}
%\date{\currenttime \ \today} % delete this line to display the current date
%
%%%% BEGIN DOCUMENT
%\begin{document}
%
%\maketitle

\section{Results}

We first describe the differences in Solano turbine-level wind resulting from the soil moisture tests (Section \ref{subsec:CharWindChanges}).  We then investigate the physical mechanism linking changes in soil moisture to changes in Solano wind timing and magnitude (Section \ref{subsec:PhysMech}).

\subsection{Characterization of Solano wind sensitivity to soil moisture}
\label{subsec:CharWindChanges}

\subsubsection{Regional sensitivity}

\textit{Time series of the upper level (700 hPa) wind, which reflects the synoptic forcing, and metrics of the land surface heating (including soil moisture, surface temperature, and sensible heat flux) are shown in Figure XX.  The land surface heating is averaged over XX regions and is shown for each of the regional perturbation tests (Table XX.)   Soil moisture deviates X amount each day before being reset at midnight.  Land surface heating, as measured by surface temperature and sensible heat flux, is strongest on XX days and is \textbf{comparable?} between regional perturbation cases.  Background winds are weak on XX days and strong on XX days.}

\begin{figure}[here]
\includegraphics[width=1\textwidth]{img/solano_wind_wetbkd_dryrg_d02_level0.png}
\caption{Time series of wind speed magnitude at 60 m AGL for the d02 grid point nearest the Solano wind farm, for the wet background and dry perturbation tests.  The model spin-up period is excluded.}
\label{fig:windSol_TseriesDryRg}
\end{figure}

\begin{figure}[here]
\includegraphics[width=1\textwidth]{img/solano_wind_drybkd_wetrg_d02_level0.png}
\caption{Time series of wind speed magnitude at 60 m AGL for the d02 grid point nearest the Solano wind farm, for the dry background and wet perturbation tests.  The model spin-up period is excluded.}
\label{fig:windSol_TseriesWetRg}
\end{figure}

The time series of Solano turbine-height (60 m AGL) wind for the regional perturbation tests (Table XX), and the difference between each regional perturbation and the corresponding control case, are shown in Figures \ref{fig:windSol_TseriesDryRg} (wet background and dry regional perturbation) and \ref{fig:windSol_TseriesWetRg} (dry background and wet regional perturbation).  In all cases, there is a strong diurnal cycle in wind speed, and wind speeds differ between the control and test cases by up to 3 m/s.  The differences in wind speed are larger on days with weaker background wind speed (days XX, cf. Figure XX) than on days with stronger background wind speed (days XX, cf. Figure XX).

\begin{figure}[here]
\begin{subfigure}{0.6\textwidth}
\includegraphics[width=1\textwidth]{img/solano_controlwind_minusmean_CA0pt25_d02_level0.png}
\caption{(a)}
\end{subfigure}
\begin{subfigure}{0.6\textwidth}
\includegraphics[width=1\textwidth]{img/solano_diurnalwind_dry_regions_d02_level0.png}
\caption{(b)}
\end{subfigure}
\begin{subfigure}{0.6\textwidth}
\includegraphics[width=1\textwidth]{img/solano_diurnalwind_wet_regions_d02_level0.png}
\caption{(c)}
\end{subfigure}
\caption{(a) Overlaid diurnal cycles of wind speed magnitude minus daily mean wind speed, at 60 m AGL for the d02 grid point nearest the Solano wind farm, for the CA-0.25 control case.  (b) and (c) Diurnally averaged differences in wind speed, at 60 m AGL for the d02 grid point nearest the Solano wind farm, for (b) the wet background/dry perturbation cases and (c) the dry background/wet perturbation cases.  Shading represents one standard deviation.}
\label{fig:windSol_DiffDiurnalWetRg}
\end{figure}

Figure \ref{fig:windSol_DiffDiurnalWetRg} illustrates the diurnal cycle of the wind and the diurnal timing of changes in each regional perturbation test case.  In Figure \ref{fig:windSol_DiffDiurnalWetRg}(a), the time series of wind for all days are overlaid, with the mean wind for each day removed.  Figures \ref{fig:windSol_DiffDiurnalWetRg}(b) and (c) show the average differences in wind speed as a function of hour of day for the dry perturbation (b) and wet perturbation (c) cases.

Solano 60 m AGL winds are strongest between 20:00 and 04:00 local time and weakest between 08:00 and 13:00 local time (Figure \ref{fig:windSol_DiffDiurnalWetRg}(a)).  Dry regional perturbations on a wet background (\ref{fig:windSol_DiffDiurnalWetRg}(b)) tend to increase Solano wind speeds in the afternoon and evening; this increase is greatest for the dry Central Valley case, with an average increase of 1 m/s spanning the hours 11:00 to 21:00.  The increase is smaller in the dry Coast Range and Sierra Nevada cases (0.5-1 m/s).  The increase happens earlier (12:00-18:00) in the dry Sierra Nevada case than in the dry Coast Range case (16:00-20:00).  Importantly, this afternoon and evening increase occurs during a transitional time from weak to strong background winds (\ref{fig:windSol_DiffDiurnalWetRg}(a)); thus, the wind increases during the afternoon hours, especially in the dry Central Valley and dry Sierra Nevada cases, represent a shift to an earlier daily ramp up.  Wet regional perturbations on a dry background (\ref{fig:windSol_DiffDiurnalWetRg}(c)) have a weaker effect on Solano wind speeds.  The wet Central Valley perturbation has the strongest effect, with a decrease of Solano wind speeds of 0.5-1 m/s during the hours 10:00-18:00.

\begin{figure}[here]
\includegraphics[angle=90,origin=c,width=1\textwidth]{img/wind_map_regions2.eps}
\caption{Wind speed (color shading) and direction (vectors) and pressure (color contours) at 110 m ASL on the d02 domain.  (a)-(d) CA-0.25 control case.  (e)-(h) dryCR case, changes in wind and pressure.  (i)-(l) dryCV case, changes in wind and pressure.  (m)-(p) drySN case, changes in wind and pressure.  Top row: average for hour 06:00 for the whole run; second row: average for hour 14:00; third row: average for hour 18:00; bottom row: average for hour 00:00.}
\label{fig:windSol_WindMapsRg}
\end{figure}

Figures \ref{fig:windSol_WindMapsRg}(a)-(d) show the 110 m ASL (60 m AGL at Solano) wind and pressure averaged over hours 06:00 (a), 14:00 (b), 18:00 (c), and 00:00 (d) over all days in the CA-0.25 control simulation. Panels (e)-(p) of Figure \ref{fig:windSol_WindMapsRg} show the average difference between the test case wind and pressure and the control wind and pressure at those four times of day.  

In the control case (panels (a)-(d)), winds through the Solano pass are strong at all hours relative to winds in the San Francisco Bay and the Central Valley.  Wind flowing through the Solano pass splits when it reaches the Sierra Nevada into a northward branch and a southward branch.  The southward branch is stronger than the northward branch at 06:00 and 14:00, and the northward branch strengthens at 18:00 and 00:00.  The pressure gradient from the San Francisco Bay to the Central Valley remains strong from 14:00 to 00:00 and is weaker by 06:00.

Coast Range: large changes in northern Central Valley wind speed and pressure gradient in afternoon (14:00 and 18:00); changes at night much smaller; changes at Solano are much smaller than changes in northern Central Valley, but changes are in phase (biggest Solano changes 14:00 and 18:00).

Central Valley: in afternoon, large changes in pressure and wind throughout Central Valley, including Solano.  Large increase in pressure gradient from San Francisco Bay to Central Valley at 14:00; increased pressure gradient persists at 18:00 and to a lesser degree at 00:00.  However, at 18:00, the zone of steepest pressure change has been pushed further eastward, and the strongest winds track this band of largest pressure gradient.  Weaker changes in wind remain at Solano at 00:00 and 06:00 (quantify this statement?).  Some weakening of the winds in the southern Central Valley at 06:00.

Sierra: moderate strengthening of pressure gradient at 14:00 and 18:00.  Pressure changes are minimal by 00:00 and 06:00.  Strengthened winds through Solano pass and middle Central Valley in afternoon (14:00 and 18:00); by 18:00, bands of largest wind increases have moved northward and southward along Central Valley, again following zones of greatest pressure gradient.

The vertical profile of wind at Solano also changes in each test case.  Figure XX shows the vertical profiles of Solano $u$, $v$, and $|u|$, averaged for each hour of the day over the 14 days of the simulation.  Panel a shows the control case (CA-0.25), and panels b-d show the averaged differences between test and control winds for the three regional test cases (dryCR, dryCV, and drySN).  \textbf{Describe the changes...}

In summary, the Central Valley soil moisture influences the Solano turbine-level winds more strongly than does the Coast Range or Sierra Nevada soil moisture.  \textit{CHECK THIS: Drier soils \textbf{(in all regions but especially in the central valley?)} increase both the maximum and minimum wind speeds and shift both the ramp-up and ramp-down earlier.  Conversely, wetter soils decrease both the maximum and minimum wind speeds and shift the ramp-up and ramp-down later.  MODIFY TO INCLUDE DIFFERENCES BETWEEN REGIONAL CASES}

\subsubsection{Scaling of wind changes with Central Valley soil moisture}

Solano winds respond most markedly to changes in Central Valley soil moisture; the scaling of Solano winds with Central Valley soil moisture magnitude is shown in Figures XX (time series) and XX (summary statistics).  Drier Central Valley soil moisture causes increases in maximum and minimum Solano turbine-level winds and causes the hour of ramp-up to shift earlier.  The maximum winds at Solano are particularly sensitive in the Central Valley soil moisture range of 0.05-0.15, with the greatest increases in maximum wind speed per unit change in soil moisture when Central Valley soil is driest.  Additionally, the Solano winds show particular sensitivity to Central Valley soil moisture when the background (Coast Range and Sierra Nevada) are wetter (soil moisture = 0.2) \textbf{(CHECK THIS)}.  The Solano minimum wind speeds also increase as the Central Valley soil dries, but the wind increases per unit soil moisture change are greatest in the soil moisture range of 0.15-0.35.

%\begin{figure}[here]
%\includegraphics[width=1\textwidth]{img/solano_wind_CV_d02_level0.png}
%\caption{Time series of wind speed magnitude at 60 m AGL for the d02 grid point nearest the Solano wind farm, for a range of Central Valley soil moisture values with (a) a wet background (soil moisture = 0.2) in the Coast Range and Sierras, and (b) a dry background (soil moisture = 0.1).  The gray area denotes the model spin-up period.}
%\label{fig:windSol_TseriesWindCV}
%\end{figure}

\begin{figure}[here]
\includegraphics[width=1\textwidth]{img/shifts_CVsmois_d02.png}
\caption{Shifts in wind characteristics as a function of Central Valley soil moisture.  (a) Change in average wind speed during the maximum period (XX - XX hrs): symbol is the mean of the changes in average maximum wind speed over all days, and error bars are one standard deviation of the changes in average maximum wind speed.  (b) Change in average wind speed during the minimum period (XX - XX hrs); symbol and error bars as in (a).  (c) Change in hour of ramp-up, defined in the text; symbol and error bars as in (a).  (d) Change in hour of ramp-down, defined in the text; symbol and error bars as in (a).  \textbf{Might need to put equations in the text to clarify how these metrics are calculated.}}
\label{fig:windSol_WindShiftsCV}
\end{figure}

The daily wind ramp-up occurs earlier when the Central Valley is drier and later when the Central Valley is wetter; the ramp-up delay with a wetter Central Valley is larger when the background (Coast Range and Sierra Nevada) soil moisture is drier (0.5-2 hrs) than when it is wetter (less than 0.5 hours).  While there is large variability in the shifts in ramp-down timing, it appears that a wetter Central Valley also causes a delay in the timing of the ramp-down.

\subsubsection{Scaling of wind changes with area modified (??)}

\subsection{Physical mechanism}
\label{subsec:PhysMech}

Next we investigate the physical mechanism by which changes in soil moisture, especially in the Central Valley, influence near-surface winds at Solano.  

\subsubsection{Scaling analysis}

A scaling analysis of the terms of the momentum equation illustrates the relative importance of the momentum advection and pressure gradient terms in determining accelerations in westerly wind velocity $u$.  

\begin{equation}
\frac{\partial u}{\partial t} = -u\frac{\partial u}{\partial x} -\frac{1}{\rho} \frac{\partial p}{\partial x} - \kappa u
\end{equation}

We neglect the Coriolis term and the $y$-direction advection and pressure gradient because topographic channeling enforces nearly westerly flow in the Solano pass, and we treat friction as a linear function of wind speed.  We take 10 m/s as a typical value for $U$, 100 km as the length scale $L$ across the Solano pass from the San Francisco Bay to the Central Valley, +1 m/s as the typical $\Delta U$ from the San Francisco Bay to the Central Valley, -1 hPa as the typical pressure difference $\Delta p$ from the San Francisco Bay to the Central Valley, 1 kg/m$^3$ as the density $\rho$, and 7$\times$10$^{-5}$ s$^{-1}$ as the linear friction coefficient $\kappa$ [Zhong et al. 2004]. Then the advection term $\frac{-U^2}{L}$ is $\mathcal{O}$(-10$^{-3}$ m/s$^2$), the pressure gradient term $-\frac{1}{\rho} \frac{\Delta p}{L}$ is $\mathcal{O}$(10$^{-3}$ m/s$^2$), and the friction term $-\kappa U$ is $\mathcal{O}$(-10$^{-4}$ m/s$^2$).  Thus, the pressure gradient and momentum advection terms dominate the accelerations in $u$, and they work in opposite directions, with the pressure gradient tending to accelerate $u$ and the momentum advection tending to decelerate $u$.  During periods of greatest modeled wind acceleration or deceleration, the modeled magnitude of $\frac{\partial u}{\partial t}$ is \textbf{approximately 5$\times$10$^{-4}$ to 10$^{-3}$ m/s$^2$ WHICH IS IT? ($\Delta U$ = 4 m/s, $\Delta t$ = 2 hr)}.  Thus, the balance of the pressure gradient and momentum advection terms can account for the modeled accelerations and decelerations.

\subsubsection{Pressure patterns driving wind}

The scaling analysis shows that the pressure gradient plays an important role in driving accelerations in $u$ in the Solano pass.  As such, we seek to identify the vertical level and horizontal distance at which the pressure gradient drives the winds, and we investigate how changes in the land surface energy balance affect pressure at this level and horizontal locations.  First, we linearly regress turbine-height $u$ (60 m AGL) against the horizontal pressure anomalies, as described in Section XX; the regression slopes and Pearson's r correlation coefficient for each grid point for the pressure anomalies at XX m ASL are shown in Figure \ref{fig:windSol_corrPgradUmap}.  Figure \ref{fig:windSol_corrPgradUmap} shows results for the dryCV run; results for other runs were similar and are not shown.  The boxes in Figure \ref{fig:windSol_corrPgradUmap} \textbf{(need to add these boxes to the figure)} outline regions with both large slopes and large r-values, implying that Solano winds are highly sensitive to the pressure anomalies.  When pressure in box A is higher than the rest of the domain, 60 m AGL winds at Solano are faster (positive slopes and r-values); in contrast, when pressure in box B is lower than the rest of the domain, 60 m AGL winds at Solano are faster (negative slopes and r-values.)

\begin{figure}[here]
\includegraphics[width=1\textwidth]{img/corr_u_p-baybox_lev250_lag1.png}
\caption{Relationship between Solano wind speed and the pressure gradient from each grid point to box A, with wind speed lagging pressure gradient by 30 min, for the dryCV run, domain d02.  (a) Linear regression slope between wind speed at the Solano wind farm (purple diamond) and pressure gradient between each grid point and the average pressure in the north San Francisco Bay region (box A); (b) Pearson's r correlation coefficient for the linear regression between each grid point's pressure gradient relative to box A and the Solano wind speed.  \textbf{Need to remove the bottom two panels}}
\label{fig:windSol_corrPgradUmap}
\end{figure}

Winds at Solano lag the pressure gradient between box A and box B (using the average pressure for each box at a given level), with correlation peaking at a lag of XX hours (Figure \ref{fig:windSol_lagcorrPgradU}).  The sensitivity (as measured by the linear regression slope) and the correlation are maximum for the 250 m ASL pressure gradient, although the correlation and sensitivity are strong for pressure gradients at all levels below 350 m ASL.  Figure \ref{fig:windSol_lagcorrPgradU} shows results for the CA-0.2 run only; results for other runs are similar, although the peak lag varies from XX in the XX run to XX in the XX run.  \textbf{Discuss sensitivity to higher (550 m) p-grad, but at a longer lag and with poorer correlation.  What does this mean?}

\begin{figure}[here]
\includegraphics[width=1\textwidth]{img/lag_corr_p_u_CA0pt2.png}
\caption{Lag correlation between the pressure gradient from box A to boxB (leading) and Solano wind speed (lagging), for the CA-0.2 run, domain d02.  (a) Linear regression slope between wind speed pressure gradient as a function of lag hours, for different vertical levels of the pressure gradient; (b) Pearson's r correlation coefficient as a function of lag hours, for different vertical levels of the pressure gradient.  \textbf{Need to remove the left two panels, only show d02.  Also need to remove the second level wind lines.}}
\label{fig:windSol_lagcorrPgradU}
\end{figure}

\subsubsection{Local heating and advective controls on pressure gradient}

\textbf{Discuss how temperature changes are known to affect pressure (summarizing arguments from sea breeze review paper.  This may need to be moved to intro or discussion, and could reference it here.)}

Air temperature at XX m depends on air temperature 2 m above the surface; the dependence is linear during the day, but at night, the air at 250 m stays warmer than the air at 2 m, as expected from radiative cooling of the land surface and resulting stabilization and suppressed mixing of the lower atmosphere.  Air temperature at XX m in box B is sensitive to near-surface air temperature in XX regions, as measured by linear regression slopes and r-values (Figure XX).  Temperature at 2 m, in turn, depends strongly on surface skin temperature, with large linear regression slopes and r-values (Figure XX).  Finally, surface skin temperature depends greatly on the soil-moisture-dependent partitioning of land surface energy fluxes between evapotranspiration and sensible heat; Figure XX shows that the drier XX case has much higher surface skin temperature at midday in the Central Valley than does the wetter XX case.

\subsubsection{Summary of mechanism}

Figure XX summarizes the changes in land surface heating, air temperature, and pressure that lead to the changes in Solano winds, for XX cases.  In the drier cases, surface temperature (increases earlier?) and reaches midday peaks that are XX to XX deg C higher than the control (or wetter) cases.  This leads to 2 m air temperatures that are approximately XX deg C higher (and any shift in timing?), and to XX m air temperatures that are XX deg C higher (and any shift in timing?).  Pressure at box A decreases by XX and at box B decreases by XX, so that the pressure gradient (is lower during the day?  shifts in timing?).  As a result, winds at Solano, which lag pressure gradient by approximately XX hrs, increase by XX m/s at XX times, and the ramp-up and ramp-down periods (both shift earlier, following the pressure gradient forcing.)

\textbf{Still to flesh out: Investigation of controls on air temperature at the important level and the important horizontal area, using energy balance.  Times when advection vs surface heating vs entrainment matter more.  How sensitive each of these components are to the change in soil moisture.}


%\end{document}