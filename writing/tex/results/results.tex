
%\documentclass[12pt]{amsart}
%\usepackage{geometry} % see geometry.pdf on how to lay out the page. There's lots.
%\usepackage{datetime}
%\usepackage{setspace}
%\doublespacing
%\geometry{a4paper} % or letter or a5paper or ... etc
%% \geometry{landscape} % rotated page geometry
%
%% See the ``Article customise'' template for come common customisations
%
%\title{Wind chapter}
%\author{Percy Link}
%\date{\currenttime \ \today} % delete this line to display the current date
%
%%%% BEGIN DOCUMENT
%\begin{document}
%
%\maketitle

\section{Results}

We first describe the differences in Solano turbine-level wind resulting from the soil moisture tests (Section \ref{subsec:CharWindChanges}).  We then investigate the physical mechanism linking changes in soil moisture to changes in Solano wind timing and magnitude (Section \ref{subsec:PhysMech}).

\subsection{Characterization of Solano wind sensitivity to soil moisture}
\label{subsec:CharWindChanges}

\subsubsection{Regional sensitivity}

The time series of Solano turbine-height wind for the regional test cases (CA-0.25, dryCR, dryCV, drySN, CA-0.1, wetCR, wetCV, and wetSN), and the difference between each regional perturbation and the corresponding control case, are shown in Figure \ref{fig:windSol_TseriesWindRg}.  In all cases, there is a strong diurnal cycle in wind speed, with strongest winds between XX and XX local time and weakest winds between XX and XX local time.  Wind speeds differ between the test cases by up to XX m/s.  

\textit{Figure XX(a) shows the average 100 m ASL wind at a peak time (hour XX) over all days in the control simulation (CA-0.25); panels b-d of Figure XX show the average difference between the test case wind and the control wind at XX hour.  Figure XX(e) shows average 100 m ASL wind at a low time (hour XX), and panels f-h of Figure XX show the average difference between the test case wind and control wind at hour XX.  At hour XX when control winds are strong, winds increase in XX regions in the XX cases but not in the XX cases.  At hour XX when control winds are weak, winds increase in XX regions in the XX cases but not in the XX cases.}

The vertical profile of wind at Solano also changes in each test case.  Figure XX shows the vertical profiles of Solano $u$, $v$, and $|u|$, averaged for each hour of the day over the 14 days of the simulation.  Panel a shows the control case (CA-0.25), and panels b-d show the averaged differences between test and control winds for the three regional test cases (dryCR, dryCV, and drySN).  \textbf{Describe the changes...}

The changes in magnitude and timing of turbine-level Solano wind speed for the regional test cases are quantified in Figure \ref{fig:windSol_WindShiftsRg} following the procedure in Section \ref{subsubsec:WindStats}.  Perturbing the Central Valley soil moisture (cases dryCR and wetCR) produces the largest changes in both maximum and minimum Solano turbine-level wind speeds.  When the Central Valley is dry (soil moisture = 0.1) on a wet background (soil moisture = 0.25), both maximum and minimum Solano winds are faster by approximately 0.5 m/s compared to the all-wet control.  Conversely, when the Central Valley is wet on a dry background, the maximum and minimum Solano winds are both slower than the all-dry control.  In both the wet and dry background cases, the changes in the minimum wind speeds are greater than the changes in the maximum wind speed.

\begin{figure}[here]
\includegraphics[width=1\textwidth]{img/solano_wind_rg_d02_level0.png}
\caption{Time series of wind speed magnitude at 60 m AGL for the d02 grid point nearest the Solano wind farm, for (a) the dry background and wet perturbation tests, and (b) wet background and dry perturbation tests.  The gray area denotes the model spin-up period.}
\label{fig:windSol_TseriesWindRg}
\end{figure}

\begin{figure}[here]
\includegraphics[width=1\textwidth]{img/shifts_regions_d02.png}
\caption{Shifts in wind characteristics in soil moisture test cases, with perturbed region indicated on the x-axis.  (a) Change in average wind speed during the maximum period (XX - XX hrs): symbol is the mean of the changes in average maximum wind speed over all days, and error bars are one standard deviation of the changes in average maximum wind speed.  (b) Change in average wind speed during the minimum period (XX - XX hrs); symbol and error bars as in (a).  (c) Change in hour of ramp-up, defined in the text; symbol and error bars as in (a).  (d) Change in hour of ramp-down, defined in the text; symbol and error bars as in (a).  \textbf{Might need to put equations in the text to clarify how these metrics are calculated.}}
\label{fig:windSol_WindShiftsRg}
\end{figure}

The changes in wind speed resulting from perturbed Coast Range or Sierra Nevada soil moisture are smaller than the changes in the perturbed Central Valley cases.  The perturbed Coast Range soil moisture affects the maximum Solano wind, again by increasing the maximum in the wet background/dry perturbation case and decreasing the maximum in the dry background/wet perturbation case; there is no change in the minimum Solano wind in the perturbed Coast Range case.  In contrast, the perturbed Sierra Nevada case shows changes in the minimum wind speed but not the maximum wind speed; again, the wet background/dry perturbation case increases the minimum wind speed, and the dry background/wet perturbation case decreases the minimum wind speed.

Figure \ref{fig:windSol_WindShiftsRg} also shows shifts in the timing of the afternoon ramp-up (panel c) and the early morning ramp-down (panel d) of wind speed.  All three regional cases show a larger shift in the hour of ramp-up than in the hour of ramp-down.  In the dry background/wet perturbation cases, the ramp-up shifts later (by approximately 0.5 to 1.5 hours), while in the wet background/dry perturbation cases, the ramp-up shifts earlier (by 0.5 to 1 hours).

In summary, the Central Valley soil moisture influences the Solano turbine-level winds more strongly than does the Coast Range or Sierra Nevada soil moisture.  \textit{CHECK THIS: Drier soils \textbf{(in all regions but especially in the central valley?)} increase both the maximum and minimum wind speeds and shift both the ramp-up and ramp-down earlier.  Conversely, wetter soils decrease both the maximum and minimum wind speeds and shift the ramp-up and ramp-down later.  MODIFY TO INCLUDE DIFFERENCES BETWEEN REGIONAL CASES}

\subsubsection{Scaling of wind changes with Central Valley soil moisture}

Solano winds respond most markedly to changes in Central Valley soil moisture; the scaling of Solano winds with Central Valley soil moisture magnitude is shown in Figures XX (time series) and XX (summary statistics).  Drier Central Valley soil moisture causes increases in maximum and minimum Solano turbine-level winds and causes the hour of ramp-up to shift earlier.  The maximum winds at Solano are particularly sensitive in the Central Valley soil moisture range of 0.05-0.15, with the greatest increases in maximum wind speed per unit change in soil moisture when Central Valley soil is driest.  Additionally, the Solano winds show particular sensitivity to Central Valley soil moisture when the background (Coast Range and Sierra Nevada) are wetter (soil moisture = 0.2) \textbf{(CHECK THIS)}.  The Solano minimum wind speeds also increase as the Central Valley soil dries, but the wind increases per unit soil moisture change are greatest in the soil moisture range of 0.15-0.35.

\begin{figure}[here]
\includegraphics[width=1\textwidth]{img/solano_wind_CV_d02_level0.png}
\caption{Time series of wind speed magnitude at 60 m AGL for the d02 grid point nearest the Solano wind farm, for a range of Central Valley soil moisture values with (a) a wet background (soil moisture = 0.2) in the Coast Range and Sierras, and (b) a dry background (soil moisture = 0.1).  The gray area denotes the model spin-up period.}
\label{fig:windSol_TseriesWindCV}
\end{figure}

\begin{figure}[here]
\includegraphics[width=1\textwidth]{img/shifts_CVsmois_d02.png}
\caption{Shifts in wind characteristics as a function of Central Valley soil moisture.  (a) Change in average wind speed during the maximum period (XX - XX hrs): symbol is the mean of the changes in average maximum wind speed over all days, and error bars are one standard deviation of the changes in average maximum wind speed.  (b) Change in average wind speed during the minimum period (XX - XX hrs); symbol and error bars as in (a).  (c) Change in hour of ramp-up, defined in the text; symbol and error bars as in (a).  (d) Change in hour of ramp-down, defined in the text; symbol and error bars as in (a).  \textbf{Might need to put equations in the text to clarify how these metrics are calculated.}}
\label{fig:windSol_WindShiftsCV}
\end{figure}

The daily wind ramp-up occurs earlier when the Central Valley is drier and later when the Central Valley is wetter; the ramp-up delay with a wetter Central Valley is larger when the background (Coast Range and Sierra Nevada) soil moisture is drier (0.5-2 hrs) than when it is wetter (less than 0.5 hours).  While there is large variability in the shifts in ramp-down timing, it appears that a wetter Central Valley also causes a delay in the timing of the ramp-down.

\subsubsection{Scaling of wind changes with area modified (??)}

\subsection{Physical mechanism}
\label{subsec:PhysMech}

Next we investigate the physical mechanism by which changes in soil moisture, especially in the Central Valley, influence near-surface winds at Solano.  

\subsubsection{Scaling analysis}

A scaling analysis of the terms of the momentum equation illustrates the relative importance of the momentum advection and pressure gradient terms in determining accelerations in westerly wind velocity $u$.  

\begin{equation}
\frac{\partial u}{\partial t} = -u\frac{\partial u}{\partial x} -\frac{1}{\rho} \frac{\partial p}{\partial x} - \kappa u
\end{equation}

We neglect the Coriolis term and the $y$-direction advection and pressure gradient because topographic channeling enforces nearly westerly flow in the Solano pass, and we treat friction as a linear function of wind speed.  We take 10 m/s as a typical value for $U$, 100 km as the length scale $L$ across the Solano pass from the San Francisco Bay to the Central Valley, +1 m/s as the typical $\Delta U$ from the San Francisco Bay to the Central Valley, -1 hPa as the typical pressure difference $\Delta p$ from the San Francisco Bay to the Central Valley, 1 kg/m$^3$ as the density $\rho$, and 7$\times$10$^{-5}$ s$^{-1}$ as the linear friction coefficient $\kappa$ [Zhong et al. 2004]. Then the advection term $\frac{-U^2}{L}$ is $\mathcal{O}$(-10$^{-3}$ m/s$^2$), the pressure gradient term $-\frac{1}{\rho} \frac{\Delta p}{L}$ is $\mathcal{O}$(10$^{-3}$ m/s$^2$), and the friction term $-\kappa U$ is $\mathcal{O}$(-10$^{-4}$ m/s$^2$).  Thus, the pressure gradient and momentum advection terms dominate the accelerations in $u$, and they work in opposite directions, with the pressure gradient tending to accelerate $u$ and the momentum advection tending to decelerate $u$.  During periods of greatest modeled wind acceleration or deceleration, the modeled magnitude of $\frac{\partial u}{\partial t}$ is \textbf{approximately 5$\times$10$^{-4}$ to 10$^{-3}$ m/s$^2$ WHICH IS IT? ($\Delta U$ = 4 m/s, $\Delta t$ = 2 hr)}.  Thus, the balance of the pressure gradient and momentum advection terms can account for the modeled accelerations and decelerations.

\subsubsection{Pressure patterns driving wind}

The scaling analysis shows that the pressure gradient plays an important role in driving accelerations in $u$ in the Solano pass.  As such, we seek to identify the vertical level and horizontal distance at which the pressure gradient drives the winds, and we investigate how changes in the land surface energy balance affect pressure at this level and horizontal locations.  First, we linearly regress turbine-height $u$ (60 m AGL) against the horizontal pressure anomalies, as described in Section XX; the regression slopes and Pearson's r correlation coefficient for each grid point for the pressure anomalies at XX m ASL are shown in Figure \ref{fig:windSol_corrPgradUmap}.  Figure \ref{fig:windSol_corrPgradUmap} shows results for the dryCV run; results for other runs were similar and are not shown.  The boxes in Figure \ref{fig:windSol_corrPgradUmap} \textbf{(need to add these boxes to the figure)} outline regions with both large slopes and large r-values, implying that Solano winds are highly sensitive to the pressure anomalies.  When pressure in box A is higher than the rest of the domain, 60 m AGL winds at Solano are faster (positive slopes and r-values); in contrast, when pressure in box B is lower than the rest of the domain, 60 m AGL winds at Solano are faster (negative slopes and r-values.)

\begin{figure}[here]
\includegraphics[width=1\textwidth]{img/corr_u_p-baybox_lev250_lag1.png}
\caption{Relationship between Solano wind speed and the pressure gradient from each grid point to box A, with wind speed lagging pressure gradient by 30 min, for the dryCV run, domain d02.  (a) Linear regression slope between wind speed at the Solano wind farm (purple diamond) and pressure gradient between each grid point and the average pressure in the north San Francisco Bay region (box A); (b) Pearson's r correlation coefficient for the linear regression between each grid point's pressure gradient relative to box A and the Solano wind speed.  \textbf{Need to remove the bottom two panels}}
\label{fig:windSol_corrPgradUmap}
\end{figure}

Winds at Solano lag the pressure gradient between box A and box B (using the average pressure for each box at a given level), with correlation peaking at a lag of XX hours (Figure \ref{fig:windSol_lagcorrPgradU}).  The sensitivity (as measured by the linear regression slope) and the correlation are maximum for the 250 m ASL pressure gradient, although the correlation and sensitivity are strong for pressure gradients at all levels below 350 m ASL.  Figure \ref{fig:windSol_lagcorrPgradU} shows results for the CA-0.2 run only; results for other runs are similar, although the peak lag varies from XX in the XX run to XX in the XX run.  \textbf{Discuss sensitivity to higher (550 m) p-grad, but at a longer lag and with poorer correlation.  What does this mean?}

\begin{figure}[here]
\includegraphics[width=1\textwidth]{img/lag_corr_p_u_CA0pt2.png}
\caption{Lag correlation between the pressure gradient from box A to boxB (leading) and Solano wind speed (lagging), for the CA-0.2 run, domain d02.  (a) Linear regression slope between wind speed pressure gradient as a function of lag hours, for different vertical levels of the pressure gradient; (b) Pearson's r correlation coefficient as a function of lag hours, for different vertical levels of the pressure gradient.  \textbf{Need to remove the left two panels, only show d02.  Also need to remove the second level wind lines.}}
\label{fig:windSol_lagcorrPgradU}
\end{figure}

\subsubsection{Local heating and advective controls on pressure gradient}

\textbf{Discuss how temperature changes are known to affect pressure (summarizing arguments from sea breeze review paper.  This may need to be moved to intro or discussion, and could reference it here.)}

Air temperature at XX m depends on air temperature 2 m above the surface; the dependence is linear during the day, but at night, the air at 250 m stays warmer than the air at 2 m, as expected from radiative cooling of the land surface and resulting stabilization and suppressed mixing of the lower atmosphere.  Air temperature at XX m in box B is sensitive to near-surface air temperature in XX regions, as measured by linear regression slopes and r-values (Figure XX).  Temperature at 2 m, in turn, depends strongly on surface skin temperature, with large linear regression slopes and r-values (Figure XX).  Finally, surface skin temperature depends greatly on the soil-moisture-dependent partitioning of land surface energy fluxes between evapotranspiration and sensible heat; Figure XX shows that the drier XX case has much higher surface skin temperature at midday in the Central Valley than does the wetter XX case.

\subsubsection{Summary of mechanism}

Figure XX summarizes the changes in land surface heating, air temperature, and pressure that lead to the changes in Solano winds, for XX cases.  In the drier cases, surface temperature (increases earlier?) and reaches midday peaks that are XX to XX deg C higher than the control (or wetter) cases.  This leads to 2 m air temperatures that are approximately XX deg C higher (and any shift in timing?), and to XX m air temperatures that are XX deg C higher (and any shift in timing?).  Pressure at box A decreases by XX and at box B decreases by XX, so that the pressure gradient (is lower during the day?  shifts in timing?).  As a result, winds at Solano, which lag pressure gradient by approximately XX hrs, increase by XX m/s at XX times, and the ramp-up and ramp-down periods (both shift earlier, following the pressure gradient forcing.)

\textbf{Still to flesh out: Investigation of controls on air temperature at the important level and the important horizontal area, using energy balance.  Times when advection vs surface heating vs entrainment matter more.  How sensitive each of these components are to the change in soil moisture.}


%\end{document}