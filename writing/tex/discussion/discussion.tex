
\documentclass[12pt]{amsart}
\usepackage{geometry} % see geometry.pdf on how to lay out the page. There's lots.
\usepackage{datetime}
\usepackage{setspace}
\doublespacing
\geometry{a4paper} % or letter or a5paper or ... etc
% \geometry{landscape} % rotated page geometry

% See the ``Article customise'' template for come common customisations

\title{Wind chapter}
\author{Percy Link}
\date{\currenttime \ \today} % delete this line to display the current date

%%% BEGIN DOCUMENT
\begin{document}

\maketitle

\section{Discussion}

Importance of soil moisture, sensitivity of forecasts
Mechanism
How to use soil moisture, or related observables, to improve forecasts
Caveats/uncertainties about this study

For discussion:
\begin{itemize}
\item Note other potential wind regions affected, e.g. how far into the ocean the effects extend (although maybe there's not a whole lot of offshore wind potential in CA anyway) or Altamont (with caveats about resolution and complex topography)
\item Effect of letting SMOIS evolve, vs holding fixed
\item Importance of land cover, for albedo and plant transpiration dynamics and roughness; results would have been different if we made land cover uniform, and results are dependent on accuracy of land cover parameterizations
\item This is a prototype; more complete analysis would use longer time period covering more synoptic conditions, 
\item Probable sensitivity to PBL scheme, resolution - these questions are addressed in other studies, and an analysis for operational purposes at a given site would need to optimize over these parameters
\item Would be good to run ensembles with perturbed forcing
\item How each change to smois changes circulation, and how this affects winds at a point
\item Additionally, it would be great to find an observable intermediate variable that could detect such soil moisture changes and integrate them into wind forecasts at Solano.  Measure pressure gradient between the North Bay and Sacramento?  Measure surface skin temperature?  (And potentially assimilate that to improve model soil moisture estimate?)
\item Speculate on why Solano winds are more sensitive to CV than CR or SN soil moisture.  More direct impact on the pressure gradient that drives Solano winds?
\item Discuss why the shifts in maximum vs minimum winds, why the shift earlier vs. later and changes in either ramp-up or ramp-down or both.
\item Frame the test cases: regular precip year vs dry year (governs the soil moisture of the unmanaged mountainous regions), and different levels of irrigation in the central valley.  Obviously more irrigated area in these cases than in reality - would expect sensitivity to actual irrigation to be lower.  But, still important to get the soil moisture of the large region of the central valley correct, regardless of whether it's irrigated.
\item Specific results are particular to the Noah LSM's representation of soil water and evapotranspiration.  Issues with this representation, and difficulty mapping soil moisture values to values measured in the field (do I have any evidence to back this up?)
\item What do I want wind forecasters to take away: wind in NWP models is sensitive to soil moisture, so it is important to get it as accurate as possible.  Calculate some estimates of forecast error in MW from soil moisture errors.  These are the types of errors you might expect from errors in soil moisture: generally, can cause shift in max and min winds, and in timing of diurnal ramp-ups and ramp-downs; specifically for the Solano case, ... .  For the Solano case, the important regions are XX.  The most sensitive soil moisture ranges are XX.  
\item The effect of soil moisture is expected to be strongest when local and regional land surface heating drives winds, i.e. during the warm season with weak to moderate synoptic winds.  In California, these conditions overlap with the season of greatest wind energy production, making soil moisture an important variable in wind energy forecasts.
\end{itemize}

\end{document}