
\documentclass[12pt]{amsart}
\usepackage{geometry} % see geometry.pdf on how to lay out the page. There's lots.
\usepackage{datetime}
\usepackage{setspace}
\doublespacing
\geometry{a4paper} % or letter or a5paper or ... etc
% \geometry{landscape} % rotated page geometry

% See the ``Article customise'' template for come common customisations

\title{Wind chapter}
\author{Percy Link}
\date{\currenttime \ \today} % delete this line to display the current date

%%% BEGIN DOCUMENT
\begin{document}

\maketitle

\section{Results}

Central valley has the biggest impact.  Time series of changes shows dry central valley on wet background increases wind by XX in peak hours and by XX in minimum hours, and shifts the afternoon ramp-up earlier by XX (Figure XX that quantifies these shifts, with error bars for std dev over the days.)  The wet central valley on a dry background decreases Solano wind by XX in peak hours and by XX in minimum hours, and shifts the afternoon ramp-up later (?) by XX (Figure XX).  The soil moisture of the Coast Range affects XXX (max winds? min winds? onset), with XX shifts in coast-dry-state-wet case, and XX shifts in the coast-wet-state-dry case.  The soil moisture of the Sierra Nevada affects XXX (max winds? min winds? onset), with XX shifts in Sierra-dry-state-wet case, and XX shifts in the Sierra-wet-state-dry case.

The surface temperature and heat flux changes associated with each of these regional test cases are shown in Figure XX.  The changes in sensible heat flux from surface to atmosphere are (similar/different) in all three regional cases.  The changes in surface temperature are (similar/different).  If similar: even though the magnitude of the forcing is similar between the three regional cases, the change in wind was greatest in the Central Valley case, suggesting that turbine-level winds at  Solano are more sensitive to Central Valley soil moisture than to soil moisture in the Coast Range or Sierra Nevada.

Because the turbine-level winds at Solano show the greatest sensitivity to Central Valley soil moisture, we focus on Central Valley soil moisture for the following tests.  We test how the changes in Solano turbine-level winds scale with Central Valley soil moisture.  Table XX shows the tests conducted: Central Valley soil moisture was set to a range of values from 0.05 to 0.35, and each Central Valley soil moisture was tested with two mountain (Coast Range and Sierra) soil moisture values (0.1 and 0.2).  The timeseries of winds at Solano are shown in Figure XX; Figure YY shows the average shift in maximum winds, minimum winds, and shift in time of ramp-up and ramp-down, as a function of Central Valley soil moisture.  The winds at Solano are particularly sensitive in the Central Valley soil moisture range of XX, and are particularly sensitive when the Coast Range and Sierra Nevada are (wetter(0.2)/drier(0.1)).

Next we investigate the physical mechanism for the response of Solano turbine-level winds to soil moisture changes in the Central Valley.  Additionally, it would be great to find an observable intermediate variable that could detect such soil moisture changes and integrate them into wind forecasts at Solano.  A scaling analysis of the terms of the momentum equation indicates that the momentum advection and pressure gradient terms dominate accelerations in u.  ADD MOMENTUM EQUATION.  We neglect the Coriolis term because topographic channeling enforces nearly westerly flow, so that v must remain small.  We take 10 m/s as a typical value for U, 100 km as the length scale L across the Solano pass from the San Francisco Bay to the Central Valley, 1 m/s as the typical difference in U between the San Francisco Bay and the Central Valley, 1 hPa as the typical pressure difference between the San Francisco Bay and the Central Valley, and XX as the eddy viscosity. Then the advection term is XX, the pressure gradient term is XX, and the friction term is XX.  This shows that (the pressure gradient drives accelerations in u in the Solano pass.  As such, we seek to identify the vertical level and horizontal distance at which the pressure gradient drives the winds, and we investigate how changes in the land surface energy balance affect pressure at this level and horizontal locations.)

Pressure gradients that correlate well with u, and dpgrad vs du - which locations and which levels?  Any lag?

Pressure tracks air temperature - which locations and which levels?  Any lag?  Discuss how temperature changes affect pressure (summarizing arguments from sea breeze review paper.  This may need to be moved to intro or discussion, and could reference it here, but worry about that later.)

Air temperature depends on surface temperature - any lag between surface and temperature at important air level?

Then track the propagation of changes from soil moisture, to surface temperature, to air temperature, to pressure gradient, to wind.  When are the changes in each piece the biggest?  Are there lags in communicating changes in one to changes in another?

Investigation of controls on air temperature at the important level and the important horizontal area, using energy balance.  Times when advection vs surface heating vs entrainment matter more.  How sensitive each of these components are to the change in soil moisture.




Characterize the changes in Solano winds between the control case (whole state dry) and the three regional test cases (Coast Range, Central Valley, and Sierra Nevada wetter).  Time series plot, and table summarizing changes in maximum and minimum daily wind speed, total rms, timing of increase and decrease.  For a chosen maximum time, show maps of changes in hub-heigh winds and pressure at some low level (determined by analysis in the next section.)

How do the changes in soil moisture in distant regions ($>$50 km?) affect the winds at Solano?
\begin{itemize}
\item Winds at Solano in control case correlate with pressures at certain places (-----) and with pressure gradient between Sacramento (in Central Valley directly across the pass from Solano) and certain places (-----).  Show a few vertical levels, and state the time lag if any.  For test cases, maps of correlation between delta wind and delta pressure.
\item State explicitly: winds at Solano depend most strongly on pressure gradients between x and y, at a vertical level of z, with a time lag of q.  In particular, note that low level pressure matters more than higher levels in this causal sequence. (Not that upper level pressure doesn't matter, but that for the experiments here, changes in pressure at low levels explain wind changes better than do changes in pressure at higher levels.)
\item How p gradient of interest changes in each of the test cases.  Time series plots, and quantify timing shifts and changes in max/min.  Also, if correlation maps change, show that and discuss... 
\item Pressure at this important level varies with temperature at low levels.  Scatter plot for Sacramento box, pressure level of interest against temperature at a range of levels, with a given lag (whatever gives the best correlation.)  Maps of correlations between delta p averaged over region of interest vs delta T at a given level.  If changes in pressure depend mostly on local changes in temperature, point that out.
\item Temperature at low levels closely tracks temperature of the surface during the day; at night, temperature of the air at ---- level decouples somewhat from the surface temperature, staying X-X degrees warmer (explained by radiative cooling of land surface stabilizing the near-surface atmosphere, inhibiting vertical mixing and limiting heat exchange between surface and air above the stable near-surface layer.)  Show this with scatter plots, colored by hour, for the region of interest.  Also scatter changes in air temperature vs changes in surface temperature.
\item Finally, surface temperature provides the link to soil moisture, because drier soil reduces evaporative cooling and increases the surface temperature, while wetter soil increases evaporative cooling and decreases the surface temperature.  Show this with time series of surface temperature in wet vs. dry case, and with map of change in surface temperature at a peak time.  (For all 3 regions or just CV?)  Any energy balance explanation needed?  Or just refer to second chapter?
\item Recap the chain of causality from changes in soil moisture to changes in winds at Solano: (determine whether to describe increase or decrease in soil moisture) - decrease soil moisture increases surface temperature, which translates quickly to increased air temperature up to XX m above ground.  The increased air temperature causes decreased pressure in this region.  Depending on the region: this changes the pressure gradient from X to X.  This pressure gradient does or does not matter for the Solano winds, depending on region.
\item These changes in temperature and pressure in the XX region have secondary effects on the atmospheric energy balance in the region by changing horizontal temperature advection and convective entrainment.  ....describe
\item How do these changes explain the changes in timing and magnitude, in particular?
\end{itemize}

Now that we understand the sequence of the mechanism, how sensitive is the mechanism to the magnitude of the soil moisture change, to the base state soil moisture, and to the areal extent of the soil moisture change?  Frame for this: hypothetical dry summer (mountains dry), with uncertain levels of irrigation (uncertainty in Central Valley soil moisture amount and areal extent.)  Test a range of soil moisture values, and three configurations of irrigation extent.

\begin{itemize}
\item plot of changes in wind vs changes in smois, for different base smois
\item plot of changes in pgradient, and changes in energy balance terms (sensible heat, advection, convective entrainment)
\item comment on how it scales
\item similar plots for agric-land-only and for middle-CV-only cases, and comment on areal dependence (patchiness in agric-only case and central vs north-south extent in middle-CV case)
\end{itemize}


\end{document}